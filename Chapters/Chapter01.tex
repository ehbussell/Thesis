% !TEX root = ../thesis.tex
%

\chapter{Introduction}\label{ch:intro}

\section{Motivation}

Plant disease generally, global trends, threats to food security, costs.

Threats to conservation, biodiversity, natural resources etc. Forest effects lack of resiliency, time scales for response.

\subsection{Management efforts}

Worldwide control efforts against plant disease, plant health risk register, quarantine efforts etc.

Large scale failures of disease control, e.g. Dutch elm disease. Citrus canker and associated costs.

Issues of surveillance/detection, how widespread the disease is once it has been identified and policy put in place. Large costs, but who should pay? Particularly a problem with large scale control of tree diseases.

\section{Model-based disease management decisions}

Increasingly models are used to predict effect of disease and inform management across human, animal and plant epidemiology. Whilst many different models are used, increasingly complex simulation type models are being used to inform policy.

\subsection{Simulation models}

Description of simulation model, and its benefits. Complex in order to capture real-world dynamics for accurate predictions of future spread.

Complexity makes it difficult to optimise, resulting in strategy testing.

Examples of use across humans, animals and plants.

In plant disease, lack of data and late detection mean by the time simulation model can be used, most models simply show that it is too late for effective widespread control. For example with SOD, ash dieback, citrus. But have little to say what should or could be done to limit the effects of disease, for example local control, or protection of high value resources.

\section{Sudden oak death}

Background to pathogen, and disease effects on trees.

Current state of epidemics in California, Oregon and UK. Different lineages and mating types.

\subsection{Control of sudden oak death}

Impossibility of widespread eradication using simulation model and strategy testing.

But local management efforts have been effective. Oregon as case study for effective local `slow the spread'. Also effectiveness of strategies at forest stand scale, seeking to restore forest health. Models, though, have said little about how these local management strategies should be carried out.

What is goal of control? Simulation models primarily used for large scale eradication goals, but how can models be used to protect local resources. Can models be used to optimise control for the specific local management goals of protecting valuable regions or hosts?

\section{Aims and scope}

Seek to develop methods for using mathematical models to optimise local SOD management. We aim to develop frameworks that can be used more generally to optimise control on complex simulation models. We will couple the accurate predictions from simulation models with mathematical results from the field of optimal control theory, elevating abstract mathematical results into practical management strategies in a framework that could be used for policy.

\section{Overview}

Breakdown of thesis. Start with optimal control theory, and how can be used to optimise time-dependent control strategies. We will apply this to a simple example case of protecting a high value region from a spreading epidemic. We will then develop a framework for coupling these optimal control results with simulation models in Chapter~\ref{ch:complex_models}, before applying these to the protection of tanoak, and the protection of Redwood National Park.