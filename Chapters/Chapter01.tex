% !TEX root = ../thesis.tex
%

\chapter{Introduction}\label{ch:intro}

\section{Motivation}

Infectious diseases of plants pose a serious economic and environmental threat across the globe. New pathogens are being introduced into novel environments at ever increasing rates, driven by increased international trade, climate change and agricultural intensification, causing significant damage to crops and natural environments \citep{anderson_emerging_2004, brasier_biosecurity_2008,}. Yield losses of \SIrange[range-units=single]{20}{30}{\percent} are seen globally across the major food crops of wheat, rice, maize, potato and soybean, with major implications for food security \citep{strange_plant_2005,oerke_crop_2006,savary_global_2019}. The economic cost from all crop losses to plant pathogens in the US has been estimated at \$33~billion USD \citep{pimentel_update_2005}.

Plant pathogens also affect wild plants in the natural environment, with increasing numbers of new diseases reported in forest ecosystems \citep{stenlid_emerging_2011}. Important current examples include ash dieback in Europe \citep[caused by \emph{Hymenoscyphus fraxineus};][]{kowalski_pathogenicity_2009, defra_tree_2014}, sudden oak death (caused by \emph{Phytophthora ramorum}) in the USA \citep{rizzo_sudden_2003} and Europe \citep{brasier_plant_2010}, olive quick decline syndrome in southern Europe \citep[caused by \emph{Xylella fastidiosa};][]{sicard_xylella_2018}, and sweet Chestnut blight in Europe \citep[caused by \emph{Cryphonectria parasitica};][]{milgroom_biological_2004}. Forests are a key part of a landscape, and provide important ecosystem services \citep{bateman_bringing_2013}. The economic value of forests is difficult to quantify, but the UK government has estimated that healthy forests contribute at least £5~billion to the UK economy per year, through forestry and social/environmental value \citep{green_future}. The long generation times of trees mean that resistance to disease develops slowly, or not at all, and so disease impacts have long-term implications \citep{boyd_consequence_2013}. The biodiversity of forests and the ecosystem services provided by trees are under severe threat from pests and diseases \citep{freer_tree_2017}, and disease management approaches that take more consideration of these services are urgently required \citep{boyd_consequence_2013}

\subsection{Management efforts}

There are a number of options available to decision makers for plant disease management, broadly grouped around four principles: exclusion, eradication, protection, and resistance \citep{maloy_plant_2005}. With ever-increasing numbers of pathogens, decisions must be made about which management methods, plant species and diseases to prioritise. Efforts have been made across the globe to help inform disease management, through schemes such as the UK Plant Health Risk Register \citep{baker_uk_2014}, and regional plant protection organisations such as the European and Mediterranean Plant Protection Organisation (EPPO) that develop and advise governments on disease management and surveillance strategies \citep{maloy_plant_2005}. But with very large numbers of plant species to protect, and increasing threats from pests and diseases, making these strategic decisions is not straightforward.

Mistakes in disease management decisions can be extremely costly, and major failures of management have been fairly common. In the UK, management of the Dutch elm disease outbreak in the 1970s was ultimately unsuccessful, with the loss of almost 30~million elm trees \citep{tomlinson_too_2010}. Initially it was believed that the epidemic would soon die out, and so the disease was left to run its course. In 1970 advice changed to recommend removing trees killed by the disease, but the need for larger scale efforts was not acknowledged until 1972. By this time disease containment would have been very expensive and unlikely to be successful, and so was not carried out. More recently, ash dieback has become established in the UK and it is now acknowledged that management is unlikely to make much difference to the long-term outcome across the country \citep{thomas_biological_2016}. The total economic cost of the ash dieback outbreak, including management costs and loss of ecosystem services, has recently been estimated at £15~billion \citep{hill_15_2019}.

The citrus canker epidemic in Florida highlights the potential costs of these failures. The most recent introduction of the disease was discovered in Florida in 1995, after which an eradication program was quickly initiated. This program removed and destroyed all citrus trees within a certain radius of a known infection. This radius was initially \SI{125}{\feet} (\SI{38}{\meter}) to remove asymptomatic trees that had been exposed to the disease. The radius was increased in 1998 to \SI{1900}{\feet} (\SI{579}{\meter}) as the initial radius did not remove enough trees to slow disease spread \citep{gottwald_citrus_2001}. In 2006, the ten year eradication program was abandoned once the disease was deemed endemic, after becoming widespread in commercial and residential citrus trees \citep{gottwald_citrus_2007}. A total of \$1 billion USD had been spent on the program.

Management of disease epidemics is most successful when the scale of control matches the scale of the epidemic \citep{gilligan_impact_2007}. Outbreaks have inherent spatial and temporal scales of spread, and control strategies that match these scales are the most effective. However, plants can be cryptically infected---where the host is infectious before symptoms appear---making estimation of future spread challenging \citep{thompson_detecting_2016}. Coupled with delays in disease reporting due to the high costs of surveillance \citep{parnell_generic_2014} as well as uncertainty surrounding rates of spread, determining this scale of management can be problematic. Also for these reasons, early detection of a new invasion is unlikely, but rapid deployment of resources is necessary for successful disease control \citep{cunniffe_optimising_2015}. Successful management is therefore costly, and so decision makers require robust decision making processes.

\section{Model-based disease management decisions}

Increasingly, mathematical models are used to predict the impacts of disease, and assess potential management across human, animal and plant diseases \citep{thompson_preface_2019}. Quantitative methods can be used to assist decision making by improved descriptive analysis of outbreaks, risk factors and response needs, as well as through forecasting and optimisation of interventions \citep{morgan_decision_2019}. As examples, models of outbreaks have informed ebola vaccination campaigns in humans \citep{bellan_statistical_2015}, animal culling during the UK foot and mouth epidemic \citep{keeling_dynamics_2001}, management of citrus canker in Florida \citep{gottwald_post_2007}, and of ash dieback in the UK \citep{defra_chalara_2013}.

\subsection{Simulation models}

Models that simulate the expected course of an epidemic and explicitly capture the effects of interventions can quantify the impact of a potential management strategy \citep{lofgren_opinion_2014}. These simulation models, as we will refer to them, are designed to accurately forecast disease progression under a number of intervention scenarios being considered by a decision maker. As a result, simulation models have become important tools for assessing policy decisions relating to real-time management responses, as well as to increased preparedness for future threats. This allows decision makers to examine `what if' scenarios, with all possible information available about disease impacts and uncertainty. However, to capture enough realism to be useful for guiding policy, simulation models must often be very complex \citep{basu_complexity_2013,savary_simulation_2014}. This complexity ensures that the simulation model incorporates the many factors impacting on patterns and rates of epidemic spread, for example spatial heterogeneity and variation in host susceptibility \citep{keeling_modeling_2008, anderson_preliminary_1986, smith_predicting_2002}.

The complexity of simulation models can limit the extent to which management can be optimised. With many possible interventions that can be combined and potentially vary in space, time or according to disease risk, it becomes computationally infeasible to unambiguously determine the optimal strategy. This problem with optimisation in high-dimensional space is known as Bellman's `curse of dimensionality' \citep{bellman_dynamic_1957}. As a result, for most simulation models the only viable option is to test a small subset of plausible management strategies. This `strategy testing' approach may be able to scan over a single parameter, but the set of strategies to test cannot span the entire space of control options. This makes it difficult to have high confidence in the best-performing strategy.

The approach is nevertheless commonly used to inform policy decisions across human, animal and plant disease management. In human health, \citet{jit_economic_2008} assess potential vaccination policies for human papillomavirus (HPV) in the UK, testing strategies that vary in vaccine coverage, age at vaccination, and whether the vaccine is given to boys as well as girls. The economic analysis carried out by \citet{jit_economic_2008} was used by the Department of Health to inform decisions about distribution of the HPV vaccine. Similarly, models of Ebola virus disease outbreaks in west Africa have been used to assess potential interventions including hygienic burial and contact-tracing \citep{pandey_strategies_2014}, and more recently vaccination strategies for health care workers \citep{robert_control_2019}. In animal epidemiology, models were used to inform the response to the 2001 foot-and-mouth disease (FMD) outbreak \citep{keeling_models_2005}. The simulation model developed \citep{keeling_dynamics_2001} was used to assess different animal culling strategies, and later potential vaccination strategies \citep{keeling_modelling_2003, tildesley_optimal_2006}. Finally, in plant disease, simulation models have been used to assess potential host removal strategies for tree diseases of citrus \citep{cunniffe_cost-effective_2014,cunniffe_optimising_2015, hyatt-twynam_risk-based_2017, adrakey_evidence-based_2017, craig_grower_2018}, and sudden oak death \citep{cunniffe_modelling_2016}.

In human disease outbreaks, for which modelling has played a prominent role, integration into the decision-making process can be slow because models are often built in reaction to ongoing epidemics \citep{rivers_using_2019}. The problem is amplified in plant disease management where limited funding, a lack of data, and poor surveillance means pathogens and their spread characteristics are almost always only identified once the disease is well-established. As a result, simulation models used in plant disease often simply show that it is too late for effective widespread eradication to remain a realistic proposition. This was the case with citrus canker in Florida \citep{gottwald_post_2007}, ash dieback in the UK \citep{defra_tree_2014}, and sudden oak death in California \citep{cunniffe_modelling_2016}. However, these models---and plant disease modellers in general---have said very little about how smaller-scale management, designed to achieve local goals rather than widespread eradication, could be made to be effective. Whilst studies might say what cannot be achieved, these simulation models have not been used to study what is still possible, for example through localised control or management to protect valuable resources.

\section{Sudden oak death}

The disease case study we will use throughout this thesis is the sudden oak death (SOD) epidemic in California. SOD is caused by the oomycete \emph{Phytophthora ramorum}, which can infect a very broad host range. Hosts most notably affected by the pathogen include oak, tanoak, larch, bay laurel and rhododendron, but over 100 plant species are susceptible to the disease \citep{grunwald_emergence_2012}. The disease effects vary depending on the host, but broadly split into two groups: lethal trunk infections and non-lethal foliar infections \citep{rizzo_sudden_2003}. In tanoak and oak species the disease causes large cankers to form on the main stem, eventually leading to tree death. On `spreader species', including rhododendron and bay laurel, \emph{P.~ramorum} can infect the host and sporulate, but this does not lead to host death. The pathogen spreads predominantly through short-distance rain splash dispersal of spores, but spores can be dispersed over longer distances by turbulent air currents, rivers and streams, or when carried by animals or human activity \citep{grunwald_emergence_2012}.

The disease was first detected in California in 1995 and has since spread widely along the West coast of the USA as shown in Figure~\ref{fig:ch1:map} \citep{rizzo_sudden_2003, meentemeyer_epidemiological_2011}. It has significantly impacted the nursery trade, and devastated populations of coast live oak and tanoak in California. SOD is currently found in areas covering over \SI{2000}{\km\squared} in California \citep{grunwald_ecology_2019}, with an estimated \$135M USD loss in property values attributed to the disease \citep{kovacs_predicting_2011}. In 2001 an isolated outbreak was identified in Curry County, Oregon, and in 2009 the pathogen was discovered in the UK where it is causing extensive mortality of larch \citep{brasier_plant_2010}. In the UK the disease is known as ramorum disease or sudden larch death. The European and North American outbreaks are caused by different lineages of the pathogen, designated NA1 and NA2 for the North American, and EU1 and EU2 for the European pathogens \citep{grunwald_emergence_2012}. In 2016 the EU1 lineage was discovered in Oregon forests, which is problematic as it is of a different mating type to the NA1 and NA2 lineages \citep{grunwald_ecology_2019}. Whilst sexual reproduction of the pathogen has not yet been observed, this could lead to the much more rapid evolution of more aggressive forms of the pathogen.

\begin{figure}
    \begin{center}
        \includegraphics{Graphics/Ch1/SpreadMap}
        \caption[Current state of SOD in California and Oregon]{The current state of SOD spread in California and Oregon. Counties in California with confirmed SOD infestations are quarantined (shown in red in \textbf{(a)}). A partial quarantine of Curry county in Oregon has been implemented, as shown in \textbf{(b)}.\label{fig:ch1:map}}
    \end{center}
\end{figure}

\subsection{Control of sudden oak death}

Multiple scales of management are possible in any attempt to control SOD spread. For protection of individual high-value trees, for example at the urban-wildland interface, protective chemicals can be applied through sprays or by trunk injection \citep{garbelotto_phosphonate_2009}. At the landscape scale, the only management that has proved effective is the removal of hosts \citep{hansen_epidemiology_2008}. Modelling work has shown that state-wide eradication of SOD in California has long been impossible \citep{cunniffe_modelling_2016}. The method used by \citet{cunniffe_modelling_2016} to demonstrate this was to use a complex spatially-explicit simulation model of SOD spread, and test a number of state-wide management strategies: an example of the strategy testing approach we introduced earlier. They showed that the most effective strategies prioritised control at the epidemic wavefront, but needed to have been implemented much earlier in the epidemic for widespread control to have been possible.

Despite eradication being unachievable, smaller-scale local management can still be beneficial. Since the Oregon outbreak was discovered in 2001, the disease has been actively managed, with \$22.7M USD spent on identifying and treating infested sites \citep{grunwald_ecology_2019}. The management has been effective at slowing the spread of the infestation and containing the disease within Curry county (Figure~\ref{fig:ch1:map}(b)), with 2028 and 2038 being the estimated years of arrival into Coos county with and without control, respectively \citep{sod_economics_assessment}. In some locations in Oregon, control has shown that local eradication, whilst difficult, is possible \citep{hansen_efficacy_2019}. These less ambitious local goals remain practically-relevant and achievable, but mathematical models have had little to say about how to deploy such management.

One reason for this may be that the objective of local management is less clear than that of eradication, and will vary depending on the wider forest management goals in each region. Local goals could involve slowing disease spread or protection of valuable resources (either particular hosts or regions), for example protecting culturally and ecologically important tanoak populations or slowing spread into national parks. These objectives must be considered alongside wider goals such as fire risk management and conservation. To date simulation models have only considered large-scale eradication goals, but how can models be used to optimise local control? What strategies should decision makers deploy to manage SOD and protect local resources?

\subsection{Models of sudden oak death}\label{sec:ch1:sod_models}

Many different models have been built to capture aspects of SOD spread. In this thesis we focus on SOD management strategies, and therefore require dynamic models that capture the drivers of disease spread into new regions and how these drivers are affected by possible management interventions. Much of SOD modelling, at least to start with, focussed on building risk maps. These maps show which areas are most likely to become infected, with the potential to be used to allocate control resources appropriately. \citet{meentemeyer_mapping_2004} used an expert informed, rule-based model to find high risk areas in California, based on weighted combinations of host distribution, temperature and moisture data. Later work by \citet{kelly_modeling_2007} compared environmental niche models like the model in \citet{meentemeyer_mapping_2004}, with other classifiers including logistic regression and support vector machines, and similar models have been used in Oregon \citep{vaclavik_mapping_2010}. All these risk models predict the chance of future spread, but not the dynamics of those invasions into new regions. These models cannot therefore be used to investigate the dynamics of disease spread, and importantly what effect control would have on disease progression.

Further development of these ecological niche models incorporated dispersal estimation into the risk mapping \citep{meentemeyer_early_2008}. This in effect increases the risk of invasion in areas close to known infestations. Whilst this still did not capture the dynamics of the system, it begins to capture these dynamic effects. Models were also being developed to model the spread of \emph{P.~ramorum} in the UK\@. Analysis of susceptible host movement in the UK nursery trade showed a similarity to small-world and scale-free networks, suggesting that identifying and targeting key nodes in the network could manage the disease more effectively \citep{pautasso_epidemiological_2008, jeger_modelling_2007}. \citet{harwood_epidemiological_2009} developed a stochastic network model to capture the full dynamics of pathogen spread across the whole of the UK\@. The simulations however, did not directly model different host species, so could not be used to model the differing effects on multiple species, nor were they fitted to data.

Larger scale models of SOD spread seek to capture invasion dynamics at the landscape scale. \citet{meentemeyer_epidemiological_2011} developed a model of SOD invasion to predict spread across California through to 2030. This model was later used to assess different control strategies \citep{cunniffe_modelling_2016}. Another similar model \citep{tonini_modeling_2018} integrates with the LANDIS-II forest simulation model \citep{scheller_design_2007}, designed to simulate forest disturbances. However for reasons of computational efficiency, as well as pragmatism in making very large scale predictions, both of these landscape scale models group host species together. In \citet{meentemeyer_epidemiological_2011} each simulation grid cell has a `host index' that captures the susceptibility and infectivity of the host composition in that cell. In LANDIS-II the disease model can only remove all hosts in a cohort of a given age in each cell. This means that small scale changes to host structure cannot be captured easily. However, both of these models capture sufficient dynamics to assess management strategies at the landscape scale.

Models of disease at the smaller scale of a forest stand are very limited in number. \citet{brown_forest_2009} use `stand reconstruction' to predict mortality within a forest stand. By looking for dead trees and symptomatic hosts in study plots, they estimate mortality rates and use these to predict future changes to stand structure. Again, dynamics are not captured here. \citet{cobb_ecosystem_2012} developed a differential equation model of SOD spread within a forest stand, capturing both invasion dynamics and differing mortality and infection rates by species and age class. In its current form this model does not include controls, but sufficient host dynamics are included that, with changes to the model, management strategies could be tested.

\section{Aims and overview of the thesis}

In this thesis, we seek to develop methods for using mathematical models to optimise local SOD management. More broadly, we aim to develop frameworks that can be used to optimise control on complex simulation models, improving on the strategy testing approach currently widely used. We will make use of the mathematical field of optimal control theory: a method for optimising time-dependent controls of dynamical systems We will couple the predictive power of simulation models with mathematical results from applying optimal control theory, elevating abstract mathematical results into practical management strategies in a framework that could be used for policy.

We begin in the next chapter with an introduction to optimal control theory, and how it can be used to optimise time-dependent control strategies. In Chapter~\ref{ch:three_patch}, we will apply this to a simple example case of protecting a high value region from a spreading epidemic, as a proxy for SOD invading an economically or culturally important region. This will demonstrate the optimisation capabilities of optimal control theory, but also its limitations for real-world practical management as a result of the simplicity of the underlying models. We will then develop frameworks for coupling such optimal control results with simulation models in Chapter~\ref{ch:complex_models}, using the predictive capabilities of simulation models to account for inaccuracies in the optimal control model. We show that model predictive control---a framework incorporating feedback between a simulation model and a simpler, approximate model---finds the best disease management strategies. The feedback framework is applied to the problem of tanoak protection in Chapters~\ref{ch:protect_tanoak_model} and \ref{ch:protect_tanoak_control}, showing that the framework provides robust strategies that limit the impacts of the worst-case scenario epidemics. In Chapter~\ref{ch:redwood}, we apply optimal control theory to the protection of a valuable region, Redwood National Park, showing how complex, spatially resolved control strategies can be identified. In the Discussion (Chapter~\ref{ch:discussion}) we explain how our frameworks could be used to improve plant disease outbreak responses, and where there are avenues for further study. Overall the work demonstrates how the frameworks we develop allow the insights of optimal control theory to be applied in a practical setting, with relevance to disease management across human, animal and plant health. Our unique contribution is to couple disease models with optimal control theory and systems engineering to find practical local management strategies for SOD.