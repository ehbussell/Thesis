% !TEX root = ../thesis.tex
%
\chapter{Applying optimal control theory to complex models}
\label{ch:apply_to_complex}

\section{Introduction}
\label{sec:ch4:Intro}

In the previous chapters we have introduced OCT and shown how it can be used to find the best disease management strategies. The mathematical complexity of finding these optimal strategies however, means major simplifications to the system as modelled are required to allow progress to be made using OCT. There is no standard for putting the results from these mathematically motivated simplifications into practice and so it is often unclear how these strategies would perform if adopted by policy makers. In this chapter we will investigate frameworks for translating OCT results into practical and realistic management policies.

Robust policy decisions require accurate predictions of future disease dynamics. Increasingly, complex simulation models incorporating detailed representations of disease transmission processes are used to assess the potential impact of a given intervention strategy. To ensure accurate epidemic predictions, simulation models designed to aid decision making must be highly complex. As discussed in Chapter 1, this often makes optimisation of control strategies infeasible, particularly when control measures can vary over time, in space or according to disease risk \citep{bellman_dynamic_2013}. For most simulation models the only viable option is then to use the model to evaluate a small subset of plausible strategies that remain fixed during the epidemic, potentially scanning over a single parameter such as a culling radius. We shall refer to this approach as `Strategy Testing'. Using this approach makes it difficult to have high confidence in the best-performing strategy, since with no framework for choosing it, the set of strategies under test is likely to be biased. Further to this, as the set to test cannot span the entire space of control options, it is unlikely that the true optimum will be found.

In this chapter we ask whether the optimisation capabilities of OCT might be combined with the accurate predictions made by simulation models, to give improved management strategies. What framework should be applied to make practical use of OCT? In Section~\ref{sec:ch4:Frameworks} we describe two methods from control systems engineering for applying OCT results to simulation models: open-loop and model predictive control. A network epidemic model is developed in Section~\ref{sec:ch4:Controlling} to showcase these frameworks, and we illustrate the potential benefit of using OCT alongside simulation models in Section~\ref{sec:ch4:Results}. We seek to answer how, under current computational constraints, results from OCT can be applied whilst maintaining the realism required for practical application.

\section{Frameworks for practical optimal control}
\label{sec:ch4:Frameworks}

Outside of epidemiology, OCT has had wider use on approximate models of complex systems. A recent study reviews the use of OCT for agent-based models (ABMs) \citep{an_optimization_2017}, a type of model that simulates the individual behaviour of autonomous agents. An \textit{et al}.\ \citep{an_optimization_2017} suggest the use of a model that approximates the dynamics of the ABM, designed to be simple enough to allow mathematical analysis of the optimal control. A suitable approximate model is chosen and fitted either to real data, or to synthetic data from the ABM. The OCT results from the approximating model are then mapped onto the ABM to be tested: a process referred to as `lifting', which could equally well apply to the detailed epidemic simulation models considered in this chapter. We now describe two possible frameworks from control systems engineering for making use of this control lifting approach.

\subsection*{Open-loop control}

The first method is the simplest application of control lifting, and the framework implicitly suggested by An \textit{et al}.\ \citep{an_optimization_2017}. Control is optimised on the approximate model once using the initial conditions of the simulation model. The resulting optimal control strategy is lifted to the simulator and applied for the full simulation run time (Figure~\ref{fig:ch4:mpc_framework}). Repeated simulation of the OCT strategy on the simulation model allows assessment against other possible control strategies. The optimisation gives a single, time dependent strategy for all simulation realisations, and so does not incorporate any feedback. It is therefore referred to as `open-loop' control, as it is fully specified by the simulation initial conditions and the trajectory predicted by the approximate model. Use in epidemiology is uncommon, although Clarke \textit{et al}.\ \citep{clarke_approximating_2013} use OCT in an approximate model to find optimal levels of Chlamydia screening and contact tracing which are then mapped onto a network simulation.

\begin{figure}[htb]
    \begin{center}
        \includegraphics[width=0.4\textwidth]{Graphics/Ch4/MPC_Framework}
        \caption{Open-loop and model predictive control (MPC). The model hierarchy is shown, with optimised controls from the approximate model directly lifted to the simulation model. The real system is in green, the models and fitting processes are in blue, and the control framework is in orange. Without the orange dashed feedback loop, this is open-loop control. MPC resets the state of the approximate model at regular update steps, before re-optimising and lifting controls to the simulation model until the next update time.}
        \label{fig:ch4:mpc_framework}
    \end{center}
\end{figure}

\subsection*{Model predictive control}

Open-loop control requires the approximate model to remain accurate over the time scale of the entire epidemic. For tractability, however, the approximate model must necessarily omit many heterogeneities present in the simulation model, such as spatial effects and risk structure. When strategies resulting from OCT are then applied to the simulation model or to the real system, the disease progress is likely to deviate systematically from the trajectory predicted by the approximate model. Model predictive control (MPC) is an optimisation technique incorporating system feedback that can take such perturbations into account \citep{camacho_model_2012, lee_model_2011}. At regular update times the values of the state variables in the approximate model are reset to match those in the simulation at that time. The control is then re-optimised and the new control strategy is applied to the simulation until the next update time. The approximate and simulation models are therefore run concurrently, with multiple optimisations per realisation, to ensure that the approximate model and control strategy closely match each individual simulation realisation (Figure~\ref{fig:ch4:mpc_framework}). These multiple optimisations are computationally costly but tractable, unlike performing optimisation on the full simulation model.

MPC has had some use within the epidemiological literature, the majority being for control of drug applications for single individuals rather than control of epidemics at the population level. Examples include finding management strategies for HIV that are robust to measurement noise and modelling errors \citep{zurakowski_model_2006, david_receding_2011}, and control of insulin delivery in patients with diabetes \citep{hovorka_nonlinear_2004}. These studies highlight the benefits of MPC for robust control, i.e.\ control that remains effective despite system perturbations. However, only one study concentrates on epidemic management \citep{selley_dynamic_2015}, and that does not explicitly test the feedback control on simulations.

\section{Controlling an epidemic spreading on a network}
\label{sec:ch4:Controlling}

To demonstrate open-loop and MPC for epidemic management we use a stochastic SIR network model including host demography and risk structure. The model is deliberately kept simple to show how the underpinning idea is broadly applicable across human, animal and plant diseases. Whilst the model and its parameters are arbitrary and do not represent a specific disease, we use it to represent a scenario in which a simulation model has already been fitted to a real disease system; the network model is therefore used here as a proxy for a potentially very detailed simulation model.

Both the open-loop and MPC frameworks require an approximate model with which optimisation using OCT is possible. In this chapter we will use two different levels of approximation to assess how this approximate model should be chosen. The resulting strategies using both open-loop and MPC will be tested against a Strategy Testing approach, where a limited number of plausible interventions that do not vary during the epidemic are tested on the simulation model. 

\subsection*{Simulation Model}

In our model, infection spreads stochastically across a network  of nodes that are clustered into three distinct regions (figure 2a). Each node contains a host population stratified into high and low risk groups. The infection can spread between individuals within nodes and between connected nodes. The net rate of infection of risk group $r$ in node $i$ is given by:
\begin{linenomath*}
    \begin{equation}
        S_i^r \sum_j \sigma_{ij} \left(\rho^{rH}I_j^H + \rho^{rL}I_j^L\right)\;,
    \label{eqn:sim_model}
    \end{equation}
\end{linenomath*}
where $S$ and $I$ are numbers of susceptible and infected hosts respectively, subscripts identify the node, and superscripts specify high ($H$) or low ($L$) risk group. The sum is over all connected nodes including the focal node itself, with the relative transmission strength into node $i$ from node $j$ given by $\sigma_{ij}$, and risk structure given by the \num{2x2} matrix $\rho$. Full details of the model are given in the supplementary material. Although not limited to these applications, the model in Equation~\ref{eqn:sim_model} could represent crop or livestock diseases spreading through farms, or sexually transmitted infections spreading through towns, cities or countries.

Mass vaccination is the only intervention we consider, with the potential to target based on both risk group and region but randomised across host infection status (i.e.\ the vaccine is given to all hosts but is only effective on susceptibles). Logistical and economic constraints are included through a maximum total vaccination rate ($\eta_{\mathrm{max}}$) that can be divided between risk groups and regions. Within each group susceptibles are vaccinated at rate: $f\eta_{\mathrm{max}}S/N$, where $f$ is the proportion of control allocated to that group, and $N$ is the total group population.

Optimal allocation of the vaccination resources minimises an epidemic cost $J$ representing the disease burden of the epidemic across all infected hosts over the simulation time ($T$): $J = \int_{t=0}^TI(t)\mathrm{d}t$. In common with the particular control we consider and the risk and spatial structures, this simple choice of objective function was made merely to illustrate our methods, but the framework generalises immediately to more complex settings.

\section{Results}
\label{sec:ch4:Results}

\section{Discussion}
\label{sec:ch4:Discussion}
