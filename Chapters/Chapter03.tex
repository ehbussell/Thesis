% !TEX root = ../thesis.tex
%
\chapter{A simple case: protecting a high value region}\label{ch:three_patch}

\section{Introduction}\label{sec:ch3:intro}

This chapter is about...

\section{Methods}

Overview of methods, patch model, control methods, and solving the optimisation problem.

\subsection{Invasion model}

Description of abstraction to patch model, with high value region and external source of inoculum from generally infested region.

Mathematical description of model and parameterisation, with controls.

\subsection{Optimal control problem}

Objective of control and budget constraint.

Possible expected outcomes. Overview of direct and indirect approaches.

\subsection{Indirect formulation}

Pontryagin maximum principle sets up a two point boundary problem; state and adjoint state. Derivation of Hamiltonian.

Hamiltonian used to derive adjoint dynamics; equations.

Reminder of FBSM algorithm; how is determination of control carried out --- minimise Hamiltonian subject to constraint.

\subsubsection{Mixed constraint}

Overview of how mixed state/control constraint could be handled fully; and demonstration that has same effect on equations.

\subsection{Direct formulation}

Description of BOCOP: inputs and algorithm/optimiser used.

\section{Results}

\subsection{Optimal strategies}

Intuitive standard strategy found.

Switching strategy description. Effect of beta on switch time --- intermediate infection rates lead to Switching.

Explain that switching is more than just FOI, i.e.\ it is not just switching when FOI internal to region 3 is larger than from region 2.

\subsection{Formulation comparison}

Concept of using switch time as metric for comparing accuracy of direct and indirect formulations. Shows that both agree, although direct finds optimum in some cases where FBSM does not.

Additional difficulties of FBSM: convergence issues. Compare time to solve problem.

\subsection{Parameter sensitivity}

Effect of parameterisation on switching time. Initial conditions and external FOI.

Effect of budget and control rates.

\subsection{Testing robustness}

Application of optimised control strategy to version of model with incorrectly parameterised infection rate. Leads to suboptimal control.

Application to stochastic version of model.

\section{Discussion}

Advantage of optimal control $\rightarrow$ finds less intuitive controls. But benefit of switching has to be balanced with addition costs of performing the switch.

General control strategies $\rightarrow$ bang-bang switching controls. Effect of getting switch time wrong, limited by simplicity of model.

Comparison of methods shows that direct formulation is more robust, and much simpler to implement. Therefore use this going forwards.

\section{Conclusions}

Switching important; direct methods better; results from simplified model do not stand up when complexity added.