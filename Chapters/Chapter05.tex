% !TEX root = ../thesis.tex
%
\chapter{Modelling tanoak decline in mixed species forest stands\label{ch:protect_tanoak_model}}

\section{Introduction\label{sec:ch5:Intro}}

Throughout this work we have developed methods for evaluating and optimising disease management. In this chapter we will begin to apply these methods to a real-world case study, the control of the \textit{Phytophthora ramorum} outbreak in the United States. We introduced \textit{P.~ramorum} and the current status of sudden oak death (SOD) epidemics in Chapter~\ref{ch:intro}, and in this chapter we will focus on building a model of SOD within individual \SI{20}{\hectare} mixed species forest stands. In the next chapter we will use this model to optimise strategies for stand level disease management. Whilst widespread eradication of SOD in California is now impossible \citep{cunniffe_modelling_2016}, localised control to slow disease spread, within a stand for example, can still be effective \citep{hansen_efficacy_2019}. Extensive control in Oregon, for example, is containing the spread within Curry county, with 2028 and 2038 being the estimated years of arrival into Coos county with and without control, respectively \citep{sod_economics_assessment}. Local controls can be particularly effective when management objectives are focussed on protection of high value resources with cultural, ecological or economic importance: the type of optimisation problem considered in Chapter~\ref{ch:three_patch}. In these cases control does not necessarily require global eradication, allowing management effort to be highly targeted.

In this chapter we will look at what management goals are appropriate for SOD spread in mixed species stands, in particular for protecting a culturally important species: tanoak. Informed by these goals, we will develop a model that captures the dynamics of tanoak decline and potential control measures against SOD. This model will be based on work by \citet{cobb_ecosystem_2012}, but adapted and reparameterised to realistically model host dynamics in a mixed species stand.

\subsection{The tanoak tree}

The tanoak, \textit{Notholithocarpus densiflorus} syn.~\textit{Lithocarpus densiflorus}, is a medium sized Californian tree related to the American chestnut. Tanoak is present along the western coast of California from near Santa Barbara up into southwestern Oregon (see Figure~\ref{fig:ch5:tanoak_range}) \citep{tappeiner_lithocarpus_1990}. Tanoak grows most effectively in humid conditions with seasonal precipitation. It grows from sea level to elevations of \SI{1500}{\meter} and can thrive on soils less suitable for conifers. However, its higher moisture requirements mean it is found more often on north slopes than south slopes \citep{tappeiner_lithocarpus_1990}. \citet{bowcutt_tanoak_2013} writes about the importance of the tanoak tree to Native American tribes in California. In this study \citeauthor{bowcutt_tanoak_2013} explains that mature trees are heavy producers of acorns, which are highly valued by indigenous tribal people. The acorns are an important source of food for northern Californian Native American tribes, as much now as historically, with only salmon eaten in greater quantities. The thick shell and high tannin content means the acorns can be stored for years, and hence for thousands of years they have been used as the basis for trade between tribes. \citeauthor{bowcutt_tanoak_2013} emphasises that the tree is used by Native American tribes for much more than just nutrition though, with fishing nets, baskets and medicines made using tanoak bark and wood.

\begin{figure}
    \begin{center}
        \includegraphics[width=0.95\textwidth]{Graphics/Ch5/Tanoak_Range}
        \caption[The tanoak tree: pictures and map of spatial range]{\textbf{(a)} Map showing the spatial range of tanoak in California and Oregon. Tanoak is found in the red areas. Data from:~\citet{plant_maps}. \textbf{(b)} Photograph of tanoak decline due to sudden oak death in Humboldt county, California from 2006. Source:~\citet{tanoak_pict2}. \textbf{(c)} Photograph of tanoak acorns, an important source of food for Native Americans. Source:~\citet{tanoak_pict}.\label{fig:ch5:tanoak_range}}
    \end{center}
\end{figure}

Tanoak is found alongside coast redwood (\textit{Sequoia sempervirens}) across coastal Californian forests, and is believed to be highly important to these ecosystems \citep{noss_redwood_2000}. As well as their cultural importance, tanoaks provide habitat and a vital winter food source for a variety of wildlife in these forests, including fishers and owls \citep{long_recent_2018}. The winter food source is particularly important given the more unpredictable and less nutritious crops from redwood trees. Damaged, and even sometimes healthy, tanoak trees resprout prolifically \citep{tappeiner_lithocarpus_1990}. This and abundant seed production could explain why tanoak is so ubiquitous alongside redwood, as it competitively excludes other species that could live in redwood forests \citep{ramage_forest_2011}. It is also this resprouting that ensures the population regenerates after the forest fires that are common in this region \citep{mcdonald_california_2002, ramage_role_2010}.

\subsection{Effects of sudden oak death}

The spread of \textit{P.~ramorum} is a significant threat to tanoak, having caused drastic declines in populations that, if continued, could lead to the extinction of this important species \citep{mcpherson_responses_2010}. Tanoak trees of all ages are highly susceptible to SOD and have a very high mortality from the disease \citep{davis_preimpact_2010}. In most other species susceptible to SOD, for example coast live oak, bole infection, where the infection spreads to the trunk, requires a secondary foliar host \citep{rizzo_phytophthora_2005}. As tanoak is the only species for which stems, twigs and foliage can all be infected, bole infection does not require such a secondary host and can therefore occur more rapidly and frequently \citep{rizzo_sudden_2003}. Tanoak infected with \textit{P.~ramorum} has a mortality rate of \SI{6}{\percent} per year, with infection leading to eventual tree death in at least \SI{50}{\percent} of cases \citep{mcpherson_responses_2010}. Some report net mortality is likely to approach \SI{100}{\percent} \citep{ramage_sudden_2010}. Field studies have also found that mortality increases with tree size, meaning the larger trees that produce more acorns are disproportionately affected \citep{cobb_ecosystem_2012}.

\citet{maloney_establishment_2005} use 120 study sites to track the establishment of SOD in coastal redwood forests dominated by redwood, tanoak and California bay laurel (\textit{Umbellularia californica}). Whilst redwood trees are not affected by SOD, they find that the presence of bay laurel is a key factor in the decline of tanoak due to SOD\@. \textit{P.~ramorum} can sporulate prolifically on bay trees, but the host is not killed by the disease \citep{davidson_sources_2008}. \citet{maloney_establishment_2005} state that the differing host mortalities due to SOD could lead to dramatic shifts in forest composition. These compositional changes can have far-reaching consequences and knock-on effects. For example, tree deaths due to SOD increase levels of dry wood in forests, and so greater fuel loads leading to greater risk of forest fire \citep{forrestel_disease_2015}. SOD affected forests also have fewer large trees than healthy forests, which reduces the amount of CO\textsubscript{2} captured by the forest. Management of SOD to retain larger tanoak can help manage carbon emissions \citep{twieg_reducing_2017}, and these wider reaching implications make effective disease interventions even more important.

\subsection{Predicting disease progression}

Mathematical models can be used to predict the future spread of disease, and hence inform control strategies. Much of SOD modelling, at least to start with, focussed on building risk maps. These maps show which areas are most likely to become infected, with the potential to be used to allocate control resources appropriately. \citet{meentemeyer_mapping_2004} used an expert informed, rule-based model to find high risk areas in California, based on weighted combinations of host distribution, temperature and moisture data. Later work by \citet{kelly_modeling_2007} compared environmental niche models like the model in \citet{meentemeyer_mapping_2004}, with other classifiers including logistic regression and support vector machines, and similar models have been used in Oregon \citep{vaclavik_mapping_2010}. All these risk models predict the chance of future spread, but not the dynamics of those invasions into new regions. These models cannot therefore be used to investigate the dynamics of tanoak decline, and importantly what effect control would have on disease progression.

Further development of these ecological niche models incorporated dispersal estimation into the risk mapping \citep{meentemeyer_early_2008}. This in effect increases the risk of invasion in areas close to known infestations. Whilst this still did not capture the dynamics of the system, it begins to capture these dynamic effects. Models were also being developed to model the spread of \emph{P.~ramorum} in the UK\@. Analysis of susceptible host movement in the UK nursery trade suggested a similarity to small-world and scale-free networks, suggesting that identifying and targeting key nodes in the network could manage the disease more effectively \citep{pautasso_epidemiological_2008, jeger_modelling_2007}. \citet{harwood_epidemiological_2009} developed a stochastic network model to capture the full dynamics of pathogen spread across the whole of the UK\@. The simulations however, did not directly model different host species, so could not be used to model the differing effects on multiple species, nor were they fitted to data.

Larger scale models of SOD spread seek to capture invasion dynamics at the landscape scale. \citet{meentemeyer_epidemiological_2011} developed a model of SOD invasion to predict spread across California through to 2030. This model was later used to assess different control strategies \citep{cunniffe_modelling_2016}. Another similar model \citep{tonini_modeling_2018} integrates with the LANDIS-II forest simulation model \citep{scheller_design_2007}, designed to simulate forest disturbances. However for reasons of computational efficiency, as well as pragmatism in making very large scale predictions, both of these landscape scale models group host species together. In \citet{meentemeyer_early_2008} each simulation grid cell has a `host index' that captures the susceptibility and infectivity of the host composition in that cell. In LANDIS-II the disease model can only remove all hosts in a cohort of a given age in each cell. This means that small scale changes to host structure cannot be captured easily.

Models of disease at the smaller scale of a forest stand are very limited in number. \citet{brown_forest_2009} use `stand reconstruction' to predict mortality within a forest stand. By looking for dead trees and symptomatic hosts in study plots, they estimate mortality rates and use these to predict future changes to stand structure. Again, dynamics are not captured here. \citet{cobb_ecosystem_2012} developed a differential equation model of SOD spread within a forest stand, capturing both invasion dynamics and differing mortality and infection rates by species and tanoak age class. Whilst in its current form this model does not include controls, since this is the only dynamic model at the stand level which explicitly models differences between hosts species, we use this model to investigate the optimisation of local control strategies.

\subsection{Key questions}

In this and the following chapter, we will seek to answer the following key questions:
\begin{enumerate}
    \item How can the model from \citet{cobb_ecosystem_2012} be adapted to allow optimisation of time-dependent disease management strategies?
    \item How should time-dependent controls be deployed under resource constraints to best preserve the valuable tanoak population in coastal redwood forests?
    \item How do these strategies compare with current recommended practice?
    \item How robust and reliable are these control results? In particular, how do these strategies perform when information about the epidemiological parameters and the pathogen distribution is incomplete throughout the epidemic?
\end{enumerate}
The first of these questions is the focus of the remainder of this chapter.

\newpage
\phantomsection\label{sec:ch5:CobbModel}
\section{Mixed species forest dynamics}

The epidemiological model in \citet{cobb_ecosystem_2012} was designed to investigate the medium and long term effects of sudden oak death on the host composition of mixed species, \SI{20}{\hectare} forest stands. As we will use this model structure extensively, we will refer to this model as the Cobb model from now on. The trees present in the forest stands considered are tanoak, bay laurel and redwood, which is not a host for \textit{P.~ramorum}. The authors were primarily interested in the decline of overstorey tanoak, here defined as trees over \SI{10}{\cm} d.b.h., and the initial host compositions in the stand that lead to the eventual extinction of tanoak. The model was parameterised using data from longitudinal field studies conducted over 5 years and across 110 \textit{P.~ramorum} invaded plots and 95 uninvaded plots, all of area \SI{500}{\metre\squared}. The epidemiological model is then used to assess how the forest structure in similar plots will change over the next 100 years, under scenarios with varying initial host proportions. The model is used to find a threshold initial level of tanoak in the forest, above which disease progression leads to elimination of large tanoaks. This is found to be around \SI{8}{\percent}, under a specific initial age structure. We will now describe the formulation of the model, as described in \citet{cobb_ecosystem_2012}, in more detail.

\subsection{Model description}

The model tracks the stem density dynamics of three different host species or groups: redwood, bay laurel and tanoak. The redwood group is also used to represent all species that are not susceptible to \textit{P.~ramorum} infection, which for the stands considered is predominantly coast redwood. Bay laurel is a `spreader' species that can be infected and is highly infectious, but does not itself suffer any significant effects from the disease. Tanoak however, is highly susceptible to \textit{P.~ramorum} infection and disease induced mortality is high, particularly in older and larger hosts. This age dependence, and the importance of overstorey tanoak, drove the authors to divide the tanoak class into separate age groups, in order to capture the effects of disease on the older trees. Four different age groups were created with the two oldest groups corresponding to the overstorey tanoak. This was deemed to capture the changes in susceptibility with age in enough detail, whilst also keeping the model as simple as possible. As differing effects on older trees are less important for the other hosts, and to reduce model complexity, the other host groups are not divided into age classes. The model tracks natural host demography, with natural mortality and seed recruitment rates for each host class. Recruitment depends on the amount of empty space available for seedling establishment, with each tanoak age class weighted to occupy differing amounts of space per stem. Over time tanoak hosts progress through the age classes. See Figure~\ref{fig:ch5:model_description}(a) for an overview of the different classes and possible transitions.

\begin{figure}
\centering
    \includegraphics[width=0.95\textwidth]{Graphics/Ch5/Model}
    \caption[Mixed stand model structure]{Diagram describing Cobb model host structure. \textbf{(a)} shows the possible host states and transitions. Only bay and tanoak are epidemiologically active, with the tanoak age classes grouped into small (tanoak 1 and 2) and large (tanoak 3 and 4) categories. \todo{Recovery?} \textbf{(b)} shows the spore deposition kernel described in \citet{cobb_ecosystem_2012}, with \SI{50}{\percent} of spores landing within the same cell and the other \SI{50}{\percent} spread over the neighbouring 4 cells.\label{fig:ch5:model_description}}
\end{figure}

The model is spatial with hosts positioned on a grid in square cells each of area \SI{500}{\meter\squared}. Recruitment and age transitions occur within a single cell, with density-dependence in the recruitment rates based on the available space in the cell. This corresponds to an assumption that seed dispersal between cells is negligible. This is reasonable since the heavy acorns predominantly fall under the crown of the tree \citep{tappeiner_lithocarpus_1990}. Infection dynamics are therefore the only interaction between cells. Infected hosts exert infectious pressure on susceptible hosts within the same cell and in the 4 adjacent cells. Infectious spores are distributed such that \SI{50}{\percent} land within the same cell, and the other \SI{50}{\percent} are distributed across the adjacent cells (Figure~\ref{fig:ch5:model_description}(b)). Bay and all age classes of tanoak are susceptible, and are infectious once infected without a latent period. There is little information available on latent \emph{P.~ramorum} infections, but evidence suggests progression to inoculum production happens quickly (within a few months; \cite{davidson_transmission_2005}). This level of latency is insignificant over the time period we have considered (\SI{100}{\years}).

The model is formulated as a system of ODEs resulting in 11 differential equations per cell. We use $S$ to indicate healthy hosts, $I$ to indicate infected hosts, and subscripts to indicate species (1: tanoak, 2: bay, 3: redwood), age class (1--4 where applicable), and cell location ($x$), following the notation used in \citet{cobb_ecosystem_2012} and as shown in Figure~\ref{fig:ch5:model_description}(a). The resulting equations for cell $x$ and age class $i$ are:
\begin{subequations}\label{eqn:ch5:cobb_model}
\begin{align}
    \begin{split}
        \dot{S}_{1,i,x} &= \delta_{1,i}\left[B_{1,x}E_x+r\alpha_{1,i}I_{1,i,x}\right] - d_{1,i}S_{1,i,x} - \Lambda_{1,i,x}S_{1,i,x} \\
        &{}\quad\quad+ \mu_1I_{1,i,x} + a_{i-1}S_{1,i-1,x} - a_iS_{1,i,x} \label{eqn:ch5:cobb_model_1}
    \end{split}\\
    \dot{I}_{1,i,x} &= -\alpha_{1,i}I_{1,i,x} - d_{1,i}I_{1,i,x} + \Lambda_{1,i,x}S_{1,i,x} - \mu_1I_{1,i,x} + a_{i-1}I_{1,i-1,x} - a_iI_{1,i,x} \label{eqn:ch5:cobb_model_2} \\\notag\\
    \dot{S}_{2,x} &= b_2\left(S_{2,x}+I_{2,x}\right)E_x - d_2S_{2,x} - \Lambda_{2,x}S_{2,x} + \mu_2I_{2,x} \label{eqn:ch5:cobb_model_3} \\
    \dot{I}_{2,x} &= - d_2I_{2,x} + \Lambda_{2,x}S_{2,x} - \mu_2I_{2,x}\label{eqn:ch5:cobb_model_4} \\\notag\\
    \dot{S}_{3,x} &= b_3S_{3,x}E_x - d_3S_{3,x} \label{eqn:ch5:cobb_model_5}
\end{align}
\end{subequations}
where tanoak dynamics are given by equations~\ref{eqn:ch5:cobb_model_1} and \ref{eqn:ch5:cobb_model_2}, bay dynamics by \ref{eqn:ch5:cobb_model_3} and \ref{eqn:ch5:cobb_model_4}, and redwood dynamics by \ref{eqn:ch5:cobb_model_5} (recall that all hosts in this class cannot become infected). All parameter meanings and symbols are given in Table~\ref{tab:ch5:parameters}. The delta function in equation~\ref{eqn:ch5:cobb_model_1}, $\delta_{1,i}$, is equal to one for the smallest age class, and zero otherwise. This ensures that recruitment is always to the smallest age class.

\begin{table}
    \small
    \centering
    \caption[Parameter values in the simulation model]{Parameter values used in \citet{cobb_ecosystem_2012}. Parameter values marked with an asterisk are set in model initialisation to impose dynamic equilibrium. See Section~\ref{sec:ch5:model_analysis} for a full description.\label{tab:ch5:parameters}}
    \begin{tabular}{@{}>{\raggedright}p{3cm}lll@{}}
        \toprule
        \textbf{Parameter} && \textbf{Symbol} & \textbf{Default Value} \\
        \midrule
        \multirow[t]{7}{3cm}{Infection rate} & tanoak to tanoak (\SIrange{1}{2}{\cm} d.b.h.)& $\beta_{1,1}$ & \SI{0.33}{\per\year}\\
        & tanoak to tanoak (\SIrange{2}{10}{\cm} d.b.h.)& $\beta_{1,2}$ & \SI{0.32}{\per\year}\\
        & tanoak to tanoak (\SIrange{10}{30}{\cm} d.b.h.)& $\beta_{1,3}$ & \SI{0.30}{\per\year}\\
        & tanoak to tanoak (>\SI{30}{\cm} d.b.h.)& $\beta_{1,4}$ & \SI{0.24}{\per\year}\\
        & bay to tanoak & $\beta_{12}$ & \SI{1.46}{\per\year}\\
        & bay to bay & $\beta_{2}$ & \SI{1.33}{\per\year}\\
        & tanoak to bay & $\beta_{21}$ & \SI{0.30}{\per\year}\\
        \midrule
        \multirow[t]{6}{3cm}{Natural mortality rate} & tanoak (\SIrange{1}{2}{\cm} d.b.h.) & $d_{1,1}$ & \SI{0.006}{\per\year}\\
        & tanoak (\SIrange{2}{10}{\cm} d.b.h.) & $d_{1,2}$ & \SI{0.003}{\per\year}\\
        & tanoak (\SIrange{10}{30}{\cm} d.b.h.) & $d_{1,3}$ & \SI{0.001}{\per\year}\\
        & tanoak (>\SI{30}{\cm} d.b.h.) & $d_{1,4}$ & \SI{0.032}{\per\year}\\
        & bay & $d_{2}$ & \SI{0.02}{\per\year}\\
        & redwood & $d_{3}$ & \SI{0.02}{\per\year}\\
        \midrule
        \multirow[t]{6}{3cm}{Recruitment rate} & tanoak (\SIrange{1}{2}{\cm} d.b.h.) & $b_{1,1}$ & \SI{0.0}{\per\year}\\
        & tanoak (\SIrange{2}{10}{\cm} d.b.h.) & $b_{1,2}$ & \SI{0.007}{\per\year}\\
        & tanoak (\SIrange{10}{30}{\cm} d.b.h.) & $b_{1,3}$ & \SI{0.02}{\per\year}\\
        & tanoak (>\SI{30}{\cm} d.b.h.) & $b_{1,4}$ & \SI{0.073}{\per\year}\\
        & bay & $b_{2}$ & *\\
        & redwood & $b_{3}$ & *\\
        \midrule
        \multirow[t]{4}{3cm}{Disease induced mortality rate} & tanoak (\SIrange{1}{2}{\cm} d.b.h.) & $\alpha_{1,1}$ & \SI{0.019}{\per\year}\\
        & tanoak (\SIrange{2}{10}{\cm} d.b.h.) & $\alpha_{1,2}$ & \SI{0.022}{\per\year}\\
        & tanoak (\SIrange{10}{30}{\cm} d.b.h.) & $\alpha_{1,3}$ & \SI{0.035}{\per\year}\\
        & tanoak (>\SI{30}{\cm} d.b.h.) & $\alpha_{1,4}$ & \SI{0.14}{\per\year}\\
        \midrule
        \multirow[t]{3}{3cm}{Tanoak age transition rate} & (\SIrange{1}{2}{\cm} d.b.h.) to (\SIrange{2}{10}{\cm} d.b.h.) & $a_{1,1}$ & \SI{0.142}{\per\year}\\
        & (\SIrange{2}{10}{\cm} d.b.h.) to (\SIrange{10}{30}{\cm} d.b.h.) & $a_{1,2}$ & \SI{0.2}{\per\year}\\
        & (\SIrange{10}{30}{\cm} d.b.h.) to (>\SI{30}{\cm} d.b.h.) & $a_{1,3}$ & \SI{0.05}{\per\year}\\
        \midrule
        \multirow[t]{2}{3cm}{Recovery rate} & tanoak & $\mu_1$ & \SI{0.01}{\per\year}\\
        & bay & $\mu_2$ & \SI{0.1}{\per\year}\\
        \midrule
        Recruitment suppression weight & species $i$ & $W_i$ & \num{1}\\
        & tanoak age class $i$ & $w_{1,i}$ & *\\
        \midrule
        Resprouting probability & tanoak & $r$ & \num{0.5}\\
        \midrule
        \multirow[t]{2}{3cm}{Spore proportion} & within cell & $f_0$ & \num{0.5}\\
        & between cell & $f_1$ & \num{0.125}\\
        \bottomrule
    \end{tabular}
    \end{table}

The recruitment rates are density dependent as they depend on the space available for colonisation in each cell, $E_x$. The empty space in cell $x$ is given by:
\begin{equation}
    E_x = 1 - W_1\sum_{i=1}^4w_{1,i}\left(S_{1,i,x} + I_{1,i,x}\right) - W_2\left(S_{2,x} + I_{2,x}\right) - W_3S_{3,x} \;.
\end{equation}
The species suppression weights $W_j$ give the relative area colonised, and hence unavailable for seedling recruitment, by each species per capita, which are assumed to all be equal to 1. The tanoak suppression weights $w_{1,i}$ give different relative space occupation for each age class. These suppression weights capture actual space occupied, as well as seedling suppression by other means, for example by blocking sunlight. The tanoak recruitment rate is made up of each individual recruitment from each age class, but all seedlings enter at the smallest age class. The tanoak recruitment rate, $B_{1,x}$, in equation~\ref{eqn:ch5:cobb_model_1} is given by:
\begin{equation}
    B_{1,x} = \sum_{i=1}^4b_{1,i}\left(S_{1,i,x} + I_{1,i,x}\right)
\end{equation}
so that recruitment is from all age classes, with seed production rates that are independent of infection status.

Finally we describe the force of infection terms $\Lambda$ in equations~\ref{eqn:ch5:cobb_model}:
\begin{subequations}\label{eqn:ch5:infection}
    \begin{alignat}{5}
        \Lambda_{1,i,x} &= f_0\left[\beta_{1,i}\sum_{j=1}^4I_{1,j,x} + \beta_{12}I_{2,x}\right] &+& f_1\sum_{y\in N(x)}\left[\beta_{1,i}\sum_{j=1}^4I_{1,j,y} + \beta_{12}I_{2,y}\right] \label{eqn:ch5:infection_1}\\
        \Lambda_{2,x} &= f_0\left[\beta_{21}\sum_{j=1}^4I_{1,j,x} + \beta_{2}I_{2,x}\right] &+& f_1\sum_{y\in N(x)}\left[\beta_{21}\sum_{j=1}^4I_{1,j,y} + \beta_{2}I_{2,y}\right] \label{eqn:ch5:infection_2}
    \end{alignat}
\end{subequations}
where $\beta_{12}$ is the rate of infection from bay to tanoak, $\beta_{21}$ from tanoak to bay, and $\beta_{2}$ within bay. The infection rate within tanoak is given by $\beta_{1,i}$, meaning each age class has a different susceptibility to infection from other tanoaks. Overall however, tanoak age classes do not vary in susceptibility to infection from bay, nor in the rate of infecting bay. Hence, $\beta_{12}$ and $\beta_{21}$ do not depend on age class. The parameters $f_0$ and $f_1$ give the proportion of spores deposited within and between cells respectively, where the sum over cells $N(x)$ is over the four cells adjacent to $x$.

In \citet{cobb_ecosystem_2012} and also throughout this chapter, the infection dynamics are initialised in the centre of a 20 by 20 grid (any one of the four central cells). Infection starts in the bay population, and the smallest tanoak age class, with \SI{50}{\percent} of these hosts starting infected. When all 3 species are present (bay, tanoak and redwood) the proportions are taken from those used in \citet{cobb_ecosystem_2012} for mixed species forest stands. This corresponds to \SI{40}{\percent} tanoak, \SI{16}{\percent} bay, and \SI{44}{\percent} redwood. The initial amount of empty space is found by initialising the model in dynamic equilibrium, see Section~\ref{sec:ch5:model_analysis}. The state variables track stem density in arbitrary units. The model is parameterised from the study plots in \citet{cobb_ecosystem_2012}, which have an average tanoak density of \SI{561}{\per\hectare}. The stem densities from the model can therefore be scaled such that the initial stem density is \SI{1400}{\per\hectare}.


\subsubsection{Conclusions using the model}

\citet{cobb_ecosystem_2012} used this model to predict forest structure changes over the next 100 years. They found that in forests with the spreader species bay present, overstorey tanoak (defined as the two largest age classes in the model) declines to near extinction (Figure~\ref{fig:ch5:cobb_host_change}(a)). Further to this, the authors showed that in forests with no bay present, only when tanoak makes up less than \SI{8}{\percent} of the initial stand composition does the pathogen not become established (Figure~\ref{fig:ch5:cobb_host_change}(b) and (c)). The Cobb study does not consider control strategies for managing tanoak decline, but later studies do consider this using an updated version of the model \citep{ross_soddr_2013}. This R package, called SODDr, simulates the same dynamics, but using a discrete time version of the model. The reason given for making the model discrete was to allow simpler inclusion of stochasticity and to improve model fitting, although both of these are possible with the continuous time version. The SODDr model has been used in conjunction with field studies to investigate the effectiveness of understorey vegetation thinning before and after an SOD disease outbreak \citep{cobb_resiliency_2017}. Another study used SODDr to forecast the effect of bay removal and tanoak thinning on forest conditions \citep{valachovic_forest_2017}, and other recent work has looked at the impact of putative host resistance and tolerance using SODDr \citep{cobb_promise_2019}. Whilst these controls are all either one-off interventions or natural resistance rather than the time-dependent control we will consider, this body of work indicates how the model is being used to ask questions about optimal disease management.

\begin{figure}[t]
    \begin{center}
        \includegraphics[width=0.95\textwidth]{Graphics/Ch5/Cobb_original}
        \caption[Mixed stand model baseline dynamics]{Dynamics of Cobb model, parameterised as in \citet{cobb_ecosystem_2012}. The three plots correspond to the three subplots in Figure~4 in \citet{cobb_ecosystem_2012}. \textbf{(a)} shows dynamics for a stand with bay, tanoak and redwood present, where large tanoaks decline to near extinction over 100 years. \textbf{(b)} shows the dynamics with no bay present and tanoak making up \SI{80}{\percent} of the initial composition. Here large tanoak declines but does not near extinction. In \textbf{(c)} however, with \SI{8}{\percent} tanoak, there is no decline in the large tanoak population.\label{fig:ch5:cobb_host_change}}
    \end{center}
\end{figure}

\subsection{Model analysis\label{sec:ch5:model_analysis}}

We aim to use the Cobb model to drive the open-loop and MPC optimisation frameworks developed in the previous chapter. This will help identify long term, time-dependent control strategies that conserve tanoak populations in mixed species stands. More importantly, we use the Cobb model to test how reliable and robust the frameworks are to a realistic disease scenario, testing handling of uncertainties in both parameters and observations. To ensure that our results are reliable we first analysed the Cobb model in more detail, to validate our own implementation and confirm results realistically capture the real-world spatial spread of SOD\@. To do this we tested our own Python implementation of the Cobb model versus SODDr as well as the original Berkeley Madonna (BM) code used to generate the figures in \citet{cobb_ecosystem_2012}, which was provided to us directly by the lead author. In doing this we found some unphysical features of the Cobb model which required correction before we used the model, most importantly spore deposition patterns and implausible seedling parameters.

\subsubsection{Spore deposition}

We first analyse an error in the spore deposition pattern, resulting in spores deposited equally over the source cell and adjacent cells. This affects parameters $f_0$ and $f_1$ in Table~\ref{tab:ch5:parameters}, and means more spores are deposited than produced.

The paper explains that \SI{50}{\percent} of spores fall within the same cell, and the other \SI{50}{\percent} are distributed over the four nearest neighbour cells. However, in both the BM code and the SODDr code the infectious pressure between cells is not distributed in this ratio. In the BM code the source cell and each of the four nearest neighbours receive \SI{100}{\percent} of the source cell's spores. In Equation~\ref{eqn:ch5:infection} this corresponds to both $f_0$ and $f_1$ equal to \num{1.0}, and hence a clearly unrealistic \SI{500}{\percent} of spores are deposited. In other words, an infection source cell provides the same force of infection to itself, and to each of its adjacent cells. The effect of this error is therefore to increase the spread rate, and change the shape of the dispersal kernel.

The spore deposition error is partially fixed in the SODDr implementation where $f_0$ equals 0.5 and $f_1$ equals \num{0.125} however, the between cell spores are deposited over the 8 nearest neighbours rather than 4. This leads to \SI{150}{\percent} of spores deposited, again physically impossible. The corrected deposition should use $f_0=$\num{0.5} and $f_1=$\num{0.125}, over 4 nearest neighbours. The effect of this is to significantly slow the rate of disease spread, and in turn forest composition changes, when compared with the results from the paper (Figure~\ref{fig:ch5:cobb_spatial_rate}).

In our implementation the realism of the spore deposition is improved further, by introducing an exponential dispersal kernel. The same proportion of spores (\SI{50}{\percent}) are deposited within the source cell as used in \citet{cobb_ecosystem_2012}, corresponding to $f_0=\num{0.5}$. The other \SI{50}{\percent} are distributed according to an exponential kernel with a scale parameter of \SI{10}{\meter}. The kernel is normalised so that total spore deposition across all cells is \SI{100}{\percent}. The spore proportion between cells becomes proportional to:
\begin{equation}\label{eqn:ch5:spore_kernel}
    \exp{\left(-d_{ij}/\sigma\right)}
\end{equation}
where $d_{ij}$ is the distance from source cell to target cell, and $\sigma$ is the scale parameter. The choice of \SI{10}{\meter} as a scale parameter is somewhat arbitrary, although consistent with distances of splash dispersal found for \emph{P.~ramorum} \citep{davidson_transmission_2005}. This also falls within the \SI{95}{\percent} credible interval found for short scale transmission by \citet{meentemeyer_epidemiological_2011}, although they used a Cauchy type kernel and the best estimate of the dispersal scale was \SI{20}{\meter}. We here use a smaller value to capture very local spread from splash dispersal, particularly given reports suggest that these larger scale models of SOD tend to overestimate disease spread \citep{valachovic_well_2017}.

\begin{figure}[t]
    \begin{center}
        \includegraphics[width=0.95\textwidth]{Graphics/Ch5/Cobb_corrected}
        \caption[Change in model dynamics under corrected parameters]{Change in dynamics when dispersal in the model is correctly parameterised, with realistic spore deposition. \textbf{(a)} shows the original dynamics using the implementation from \citet{cobb_ecosystem_2012}. In \textbf{(b)} the dispersal kernel is corrected, using an exponential kernel to distribute spores across cells. This leads to significantly slower dynamics unless other parameters are rescaled. \textbf{(c)} shows the same corrected dynamics, but over a time scale of \SI{300}{\years}. This shows how the pattern of dynamics is unchanged by correcting the spore deposition, but the time scale is affected.\label{fig:ch5:cobb_spatial_rate}}
    \end{center}
\end{figure}

\subsubsection{Model initialisation}

A further problem with the Cobb model is the parameterisation and initialisation of the system, which leads to unrealistically high recruitment rates and suppression weights for the smallest tanoak age class. This affects parameters $b_{1,1}$ and $w_{1,1}$ in Table~\ref{tab:ch5:parameters}.

The age specific tanoak suppression weights, $w_{1,i}$, are fixed such that one quarter of the space occupied by tanoak is allocated to each age class. With the initial age distribution used in the BM code this corresponds to the youngest age class, the seedlings, suppressing recruitment the most. This is clearly unrealistic, since seedlings occupy the least space. We choose to make the more realistic assumption that suppression scales directly with basal area. Whilst this is relatively simplistic, it does at least ensure that smaller stems occupy less physical space. In the paper the initial age distribution, empty space and recruitment rates are set by running the model over 1000 years and finding parameters that give approximate dynamic equilibrium. In the BM code this leads to the highest recruitment rate in the smallest age class. In other words, the tanoak seedlings produce the most seeds. This is despite the paper claiming that recruitment rates are found under the condition that seed production increases with age.

To fix the initialisation, we solve for the dynamic equilibrium analytically. By setting infection rates to zero, we find the initial empty space and age distribution that gives dynamic equilibrium. We use the recruitment rates from \citet{cobb_ecosystem_2012}, but set the recruitment rate of the smallest age class to zero. More specifically, in the disease-free case Equations~\ref{eqn:ch5:cobb_model_1} for tanoak in a single cell become:
\begin{subequations}
\begin{align}
    \dot{S}_{1,1} &= \left(\sum_{i=1}^4b_{1,i}S_{1,i}\right)E - d_{1,1}S_{1,1} - a_1S_{1,1}\\
    \dot{S}_{1,2} &= a_1S_{1,1} - d_{1,2}S_{1,2} - a_2S_{1,2}\\
    \dot{S}_{1,3} &= a_2S_{1,2} - d_{1,3}S_{1,3} - a_3S_{1,3}\\
    \dot{S}_{1,4} &= a_3S_{1,3} - d_{1,4}S_{1,4} \;.
\end{align}
\end{subequations}
By setting the left hand side of these equations to zero we impose dynamic equilibrium within tanoak. With the condition that $b_{1,1}$ is zero, i.e.\ the recruitment rate of seedlings, these are solved to find the empty space and $S_{1,2}$, $S_{1,3}$, and $S_{1,3}$ in terms of $S_{1,1}$:
\begin{subequations}\label{eqn:ch5:tan_init}
    \begin{align}
        E &= \frac{d_{1,1} + a_1}{b_{1,2}A_2 + b_{1,3}A_3 + b_{1,4}A_4}\\
        S_{1,2} &= A_2S_{1,1}\\
        S_{1,3} &= A_3S_{1,1}\\
        S_{1,4} &= A_4S_{1,1}
    \end{align}
    \end{subequations}
where
\begin{subequations}
\begin{align}
    A_2 &= \frac{a_1}{a_2+d_{1,2}}\\
    A_3 &= \frac{a_2A_2}{a_3+d_{1,3}}\\
    A_4 &= \frac{a_3A_3}{d_{1,4}}
\end{align}
\end{subequations}
To find $S_{1,1}$ we fix the initial proportions of each host type, with $p_1$, $p_2$ and $p_3$ representing the proportion of hosts that are tanoak, bay or redwood respectively. This gives that:
\begin{equation}
    \label{eqn:ch5:tan_0_init}
    S_{1,1} = \frac{p_1(1-E)}{1+A_2+A_3+A_4}\;.
\end{equation}
Finally, we fix the recruitment rates of bay and redwood such that those hosts are also in dynamic equilibrium. Namely:
\begin{subequations}
\begin{align}
    b_2 &= d_2 / E \\
    b_3 &= d_3 / E \;.
\end{align}
\end{subequations}

\subsubsection{Reparameterisation}\label{sec:ch5:reparameterisation}

What effects do these errors have on the model dynamics and conclusions from the 2012 paper? The dominant effect is the change in rate of spatial spread due to the error in spore deposition. This significantly affects the time scale of SOD invasion, slowing it beyond what is considered reasonable for SOD within stands (figure~\ref{fig:ch5:cobb_spatial_rate}). Whilst there is a lack of data covering spatial invasion rates at the scale of an individual forest stand, the time scales in \citet{cobb_ecosystem_2012} are broadly consistent with expectations. \citet{mcpherson_responses_2010} found infection rates within stand of around \SI{10}{\percent} per year in tanoak. Plots were \SI{90}{\percent} infected after approximately 10 years however, the plots have an average size of \SI{1234}{\meter\squared} whereas the stand we are modelling is \SI{200000}{\meter\squared}. In the original Cobb model the \SI{500}{\meter\squared} cells adjacent to the source cell reach \SI{90}{\percent} infection after approximately 8 years, consistent with the findings in \citet{mcpherson_responses_2010}. \textit{P.~ramorum} is known to spread locally via rain splash at distances of up to \SI{20}{\meter} \citep{davidson_transmission_2005}, but can be carried much further in rare long distance dispersal events \citep{meentemeyer_epidemiological_2011}. For within stand spread we will consider only the short scale spread which corresponds to the invasion reaching the edge of the modelled \SI{20}{\hectare} plot after ten to twenty years. This is again consistent with the time scale found in the original Cobb model, where infection reaches \SI{5}{\percent} at the stand edge after 15 years.

With this confidence in the broad timescales in \citet{cobb_ecosystem_2012}, we scale the infection rates in our reparameterised model with fixed dispersal and recruitment to match the original time scale of invasion found in \citet{cobb_ecosystem_2012}. We use the time at which the population of small tanoak increases above the large tanoak population to define the rate of invasion. All infection rates in the corrected model are then scaled by the same factor so that relative rates are kept the same, but the rate of invasion matches that of the original Cobb model implementation (figure~\ref{fig:ch5:inf_scaling}). This ensures that the dynamics are correct, with a realistic kernel distribution, whilst matching the generally accepted spread rates for SOD\@. Our choice for defining the time scale is arbitrary, but since the results show that the best fit is very close to the original at all times, other choices would not give very different results. We note that we also test parameter sensitivity in the next section.

A summary of all the parameter changes made based on the corrections described here is given in Table~\ref{tab:ch5:param_changes}.

\begin{figure}[h]
    \begin{center}
        \includegraphics[width=0.95\textwidth]{Graphics/Ch5/Cobb_scaled}
        \caption[Effect of model reparameterisation and infection rate scaling]{Effects of correcting parameterisation and spore deposition kernel, and matching time scales. \textbf{(a)} shows the original dynamics in \citet{cobb_ecosystem_2012}, with the cross over time labelled. The cross over time is the time until the large tanoak population falls below the small tanoak population, and is the metric we have chosen to measure the time scale of the epidemic. In \textbf{(b)} we show the corrected and scaled dynamics, with the epidemic time scale matched to that in \textbf{(a)}. This is matched by scaling the infection rates, with the effect of this scaling factor on the cross over time shown in \textbf{(c)}.\label{fig:ch5:inf_scaling}}
    \end{center}
\end{figure}

\begin{table}[h]
    \centering
    \caption[Summary of corrections to Cobb model]{Summary of changes made to Cobb model to correct unrealistic aspects of its parameterisation. The corrected spore deposition uses an exponential dispersal kernel with scale parameter, $\sigma=$\SI{10}{\meter}.\label{tab:ch5:param_changes}}
    \begin{tabular}{@{}p{3.5cm}p{2cm}p{4cm}p{3cm}@{}}
        \toprule
        \textbf{Issue} & \textbf{Parameters affected} & \textbf{Original} & \textbf{Corrected} \\
        \midrule
        Spore deposition & $f_0$ & 1.0 & 0.5 \\
        & $f_1$ & 1.0 & $\propto \exp{\left(-d_{ij}/\sigma\right)}$ \\
        &&&{\small{}(also over other cells in the landscape)}\\
        \midrule
        Recruitment suppression weights & $w_{1,i}$ & 1/4 of space for each & Space proportional to basal area \\
        \midrule
        Recruitment & $b_{1,1}$ & Chosen to give dynamic equilibrium ($\approx{}0.055$) & 0.0 \\
        \midrule
        Initial conditions & $S_{1,i}(0)$ & Chosen to give dynamic equilibrium & Equations~\ref{eqn:ch5:tan_init}--\ref{eqn:ch5:tan_0_init} \\
        \bottomrule
    \end{tabular}
\end{table}

\FloatBarrier
\subsection{Model sensitivity\label{sec:ch5:model_sensitivity}}

\citet{cobb_ecosystem_2012} state that their model is `designed to illustrate biotic factors affecting decline in tanoak populations invaded by \emph{P.~ramorum} and not to predict precise time scales for extinction'. To confirm that the Cobb model is still realistic after our reparameterisation, we test how sensitive the dynamics are to the exact parameters chosen. The aim here is to make sure that the general model dynamics, if not the exact numeric results, are unchanged for reasonable perturbations in the parameters. We will demonstrate that whilst the precise dynamics are complex, the overall trends in tanoak decline are robust to reparameterisation.

To test sensitivity we randomly perturb all parameters in Table~\ref{tab:ch5:parameters} from the base parameter set. As described before the empty space, tanoak suppression weights, initial conditions, and bay and redwood recruitment rates are calculated to give dynamic equilibrium for each reparameterisation. Using the perturbed parameter set, we run a forward simulation with no control to predict the future decline of tanoak. Each parameter is perturbed with a normally distributed error, with standard deviation equal to \SI{25}{\percent} of the parameter value. A truncated normal distribution is used to ensure that parameters are not made negative, which would be biologically unrealistic. This process is carried out for 200 perturbed parameter sets. The results are shown in Figure~\ref{fig:ch5:model_sensitivity}.

These results show that whilst there is variation in the predicted dynamics, the overall trends are highly consistent. Large tanoak is always predicted to decline significantly over the timescale considered, with smaller tanoak taking up a larger proportion of the tanoak age distribution. Bay and redwood are generally expected to increase in numbers, and certainly increase relative to the tanoak population. This sensitivity analysis confirms that the dynamics and main trend found in \citet{cobb_ecosystem_2012} are robust, despite the lack of data for accurate parameterisation.

\begin{figure}
    \begin{center}
        \includegraphics[width=0.95\textwidth]{Graphics/Ch5/Sensitivity_hosts}
        \caption[Sensitivity of model dynamics]{Sensitivity of the model dynamics to parameterisation. Solid lines show the dynamics with the baseline parameters. The dotted lines show the median from 100 random parameter perturbations, and the shaded areas show the 5\textsuperscript{th} and 95\textsuperscript{th} percentiles of the distribution. For clarity small tanoak and redwood are shown in \textbf{(a)}, and large tanoak and bay in \textbf{(b)}. In all cases there is significant decline of large tanoak. Note that whilst there is variation, the overall pattern of host compositional changes remains unchanged.\label{fig:ch5:model_sensitivity}}
    \end{center}
\end{figure}
\FloatBarrier

\section{Discussion\label{sec:ch5:discussion}}

In this chapter we have corrected, developed and analysed a stand level (\SI{20}{\hectare}) model of SOD-induced tanoak decline, based on a version originally formulated by \citet{cobb_ecosystem_2012}. We have improved the realism of the model through changes in the transmission kernel, and reparameterisation of the initial conditions and seedling recruitment rates. \emph{P.~ramorum} spreads spatially by splash dispersal. In the corrected model, the spatial transmission kernel is implemented as a negative exponential, thin-tailed kernel, as is commonly used in the literature to model fine-scale processes such as splash dispersal \citep{skelsey_pest_2013}. Whilst other models use fat-tailed kernels to capture long-distance dispersal of \emph{P.~ramorum} \citep[e.g.][]{meentemeyer_epidemiological_2011,harwood_epidemiological_2009}, these models make predictions at much larger scales than those considered here. In addition to this, as the stand-level model is deterministic, a fat-tailed kernel would lead to significantly higher, and unrealistic, rates of spatial spread unless the length scale of dispersal was highly restricted.

Our change to the kernel corrects the unrealistic spore deposition pattern in the original model. The corrections to the recruitment rates ensure that hosts in the seedling class do not produce seeds, as well as ensuring that seedlings correctly colonise less physical space than older trees. We have therefore fixed a pair of problems that were present in the original implementation. The necessity of these major corrections to the model could bring into question the validity of the rest of the model. However, the model is parameterised on data from a large and long-term network of plots, undoubtedly the most complete dataset for SOD impacts at the stand level. The parameter fits for the demographic processes are unchanged by the model corrections, and so the corrected model and parameterisation can be used with confidence. The model has also been used repeatedly in the literature \citep[e.g.][]{cobb_resiliency_2017, valachovic_forest_2017, cobb_promise_2019}, and the timescales of tanoak decline are consistent with other studies \citep{mcpherson_responses_2010, davidson_transmission_2005}, as discussed on page \pageref{sec:ch5:reparameterisation}. When coupled with both our extensive testing and correcting of the original code, we can therefore be confident in the conclusions made and the underlying model dynamics. The particular parameterisation is of little importance, since we have demonstrated that the host composition dynamics are robust to parameter perturbations.

Since the dynamics of pathogen spread are largely insensitive to the exact parameterisation, and the improved transmission kernel only significantly affects the timescale of invasion, it could be asked why these corrections to the model are necessary at all. Firstly, the corrections are necessary for completeness. The model we have implemented has more realistic parameters than the original implementation, and by not imposing an artificially severe truncation in pathogen dispersal, results in more realistic patterns of spread. More importantly though, these corrections could have an effect on the dynamics under a disease management intervention. As we will explore in more detail in the following chapter, the optimal control strategy is likely to depend on the pattern of spread \citep{hyatt-twynam_risk-based_2017} which is in turn affected by the choice of transmission kernel. The recruitment rate corrections are likely to affect the age distribution of tanoak after control strategies are carried out. Its therefore important that the underlying model is as accurate as possible before conclusions can be made about optimal control strategies.

Despite the large dataset underpinning the original publication of the model, there are data limitations that affect the model structure. There is very little data tracking the size progression of bay laurel and redwood trees. There is sufficient data to parameterise the four age classes within tanoak, but without additional data it is difficult to test whether the results are sensitive to the number of age classes implemented. The main difference between age classes is the higher pathogen-induced mortality for the oldest class. The change in this parameter through the four age classes suggests that fewer age classes would not be sufficient. It seems unlikely that additional classes would change this structure significantly, but there is little data to back this up.

Other parameters in the model are somewhat arbitrary, for example the scale of dispersal. This is again due to a lack of data, here a lack of detailed infection timings at a fine spatial scale that would allow parameterisation of a small-scale transmission kernel. Other parameters from the original model could vary between plot locations, and the justification for the relative infection rates in \citet{cobb_ecosystem_2012} is minimal. We have demonstrated that these parameterisations have little impact on the overall dynamics, but they could have an impact on interpretation. Optimal control results must be interpreted taking these uncertainties into account, and tested for robustness, if the results are to be used in the real world. However, the current parameterisation does offer a plausible baseline for which the potential impact of OCT can be assessed, in time perhaps motivating the data collection necessary for further model parameterisation.

\section{Conclusions\label{sec:ch5:conclusions}}

In this chapter we analysed and tested a model of SOD spread within a \SI{20}{\hectare} forest stand, with a focus on the decline of overstorey tanoak. The model has been reparameterised to more realistically capture spatial spread, whilst maintaining the previously fitted rate of invasion. Having justified the underlying dynamics, the model will be used to answer questions about optimising disease control in the next chapter.