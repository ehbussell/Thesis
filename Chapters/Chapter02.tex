\chapter{Optimal control theory}\label{ch:oct}

\section{Introduction}\label{sec:ch2:intro}

This chapter is about...

\section{Background}\label{sec:ch2:background}

How/why OCT developed. What OCT can do.

Brief overview of principles. Refer to others for details.

\section{OCT within epidemiology}\label{sec:ch2:oct_in_epidem}

General use in OCT - dynamic models of complex systems with clear goal for interventions.

\subsubsection{Early work \& general principles}

Sethi original work. Simple models. Bang-bang control. Switching control here or next para?

More complex and general case \citep{behncke_optimal_2000}. Still see bang-bang controls, concept of `front-loading'.

\subsubsection{Bioeconomics}

History and nature of optimisation - closely relate to economics literature. Importance of economics in epidemiology. Use of OCT to find cost-effective strategies.

To define what cost-effective means - need clearly defined and quantified objective. Importance of choosing objective in epidemiology, and effect on balancing of controls.

\subsubsection{State of the art and outstanding questions}

Added complexity in models and controls. Despite this, models must necessarily be highly simplified representations of system - can only test on more realistic models. \citet{rowthorn_optimal_2009} and testing of switching strategies.

In plant disease, space important feature. Simple metapopulation models used. More advanced lattice models. Additional testing of robustness of OCT strategies.

Spatial control a difficult feature to capture due to dimensionality. As well as previous examples, some use PDE type models, or revert to simpler optimisations and models \citep{epanchin_optimal_2012}.

\section{Optimisation methods}\label{sec:ch2:optim_methods}

Brief intro - numerical methods for solving the optimal control problem. Refer to \citep{betts_practical_2010} for overview of methods. Here describe main two classes, and implementation of each.

\subsection{Indirect formulation}

Overview of theory, mathematical form, and resulting optimisation problem. Connection to necessary conditions, and base OCT - could give  more insight due to relation to analytic solutions.

Forward-backward sweep method algorithm.

Limitations of FBSM.

\subsection{Direct formulation}

Outline of approach and difference to indirect approach.

Direct collocation algorithm.

\section{Conclusions}

Key point of chapter - types of OCT strategy, two main classes of optimisation method \& why might choose each.