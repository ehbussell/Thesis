% !TEX root = ../thesis.tex
%

\chapter{Optimising spatial strategies to protect Redwood National Park\label{ch:redwood}}

\section{Introduction}

In this chapter we will extend the optimal control methods we have developed to manage SOD in a more complex, spatial setting. So far our application of OCT has identified strategies that are non-spatial, in the case of protecting tanoak in forest stands in Chapter~\ref{ch:protect_tanoak_control}, or of limited spatial detail, e.g.\ the metapopulation models of Chapters \ref{ch:three_patch} and \ref{ch:complex_models}. Here we extend the setting of pathogen spread to a continuous landscape and show how OCT can be used to find spatially complex strategies that can more effectively control the spread of SOD\@. These optimal spatial strategies can vary in time and across 120 metapopulation cells in the landscape, allowing significantly increased spatial resolution in the control strategies. Plant disease management is most effective when the scale of control matches the scale of the epidemic \citep{gilligan_impact_2007, gilligan_epidemiological_2008, cunniffe_optimising_2015}, meaning that necessarily control must depend on the pattern of invasion. Spatially optimising control can target key locations that could link multiple regions \citep{minor_landscape_2011}, or that if infected would lead to large epidemics \citep{hyatt-twynam_risk-based_2017}. Prioritising management based on host risk can improve management of plant diseases \citep{cunniffe_modelling_2016}, but also more broadly of animal \citep{tildesley_optimal_2006} and human diseases \citep{fraser_factors_2004}. How can OCT be used to design these strategies?

As a case study, we use the 2010 SOD invasion along Redwood Creek, near to Redwood National Park in California. We will investigate how OCT can be used to identify spatial strategies that are designed to protect the national park from the impacts of SOD\@. The methodology we employ could be applied equally well to, for example: other plant diseases threatening important natural or commercial hosts, diseases threatening commercial animals or human diseases invading novel environments. With widespread control of SOD impossible \citep{cunniffe_modelling_2016}, designing the most effective strategies to protect valuable resources is essential. The strategies we identify will be compared with the control that was actually carried out in practice, a simple \SI{100}{\meter} cull radius. We demonstrate the benefit of using OCT for strategy design\footnote{All code for this chapter is available at \url{https://github.com/ehbussell/RedwoodCreekAnalysis}}.

\subsection{Redwood Creek sudden oak death outbreak}

Redwood National Park is located on the coast of northwestern California, just south of the Oregon border (Figure~\ref{fig:ch7:map}(a)). The park was established in 1968 to protect redwoods from extensive logging, and the combined Redwood National and State Parks (RNSP) now contain \SI{45}{\percent} of old-growth (never been harvested) redwood forest in California \citep{rnsp_website}. As well as redwoods, the parks preserve the ecosystem and natural biodiversity, conserving flora, fauna and natural features such as rivers and streams \citep{rnsp_website}. The area is also of importance to northwestern Californian Native Americans, with Hoopa and Yurok tribal lands nearby. Redwood Creek is a river passing through the national park, from the southern boundary up to Orick (Figure~\ref{fig:ch7:map}(b)).

In May 2010, stream baiting near Orick identified the presence of \emph{P. ramorum} for the first time, over \SI{80}{\km} north of the nearest Californian SOD infestation \citep{valachovic_novel_2013}. Stream baiting places mesh bags containing rhododendron leaves that are susceptible to infection by the pathogen in rivers and streams. The leaves are periodically tested in the laboratory for \emph{P. ramorum} presence and replaced with new leaves. The 2010 detection near Orick confirmed that the pathogen causing SOD was present, but the location of the inoculum source could have been anywhere within the \SI{80937}{\hectare} watershed of Redwood Creek \citep{valachovic_novel_2013}. Because of the importance of the region---due to the proximity of RNSP, tribal lands and USDA Forest Service lands---a large surveillance effort was carried out to identify the source, including additional stream baiting. In July 2010 one potential source was coincidentally found and confirmed as a small infestation in a residential area across several private properties in Redwood Valley (Figure~\ref{fig:ch7:map}(b)) \citep{valachovic_novel_2013}.

\begin{figure}[t]
    \begin{center}
        \makebox[\textwidth][c]{\includegraphics{Graphics/Ch7/StudyArea}}
        \caption[Redwood Creek study area and SOD outbreak]{Redwood Creek study area and SOD outbreak, with map background showing hill shading and natural vegetation colours. \textbf{(a)} shows the counties with confirmed SOD infestations in 2010. The red box in \textbf{(a)} shows the study region around the southern tip of Redwood National Park in California, shown in more detail in \textbf{(b)}. The national park is close to Yurok and Hoopa tribal lands. In 2010, SOD was detected through stream baiting of Redwood Creek near Orick. The infestation was later located at Cookson Ranch (red cross south of the national park). In 2014 the disease was found inside the national park at Bridge and Bond Creeks, shown by the red dots. The region over which simulations are carried out in this chapter is shown by the dashed red box.\label{fig:ch7:map}}
    \end{center}
\end{figure}

Subsequent control of the infestation involved collaborations between public, private and tribal land managers, and resulted in changes to legislation to offset some of the management costs for commercial landowners \citep{valachovic_novel_2013}. The management carried out removed all detected infected trees, and all tanoak and bay laurel within \SI{100}{\meter} of the infected trees \citep{valachovic_novel_2013}. This level of management was based on experimental treatments in southern Humboldt county \citep{valachovic_forest_2010, valachovic_suppression_2013} and on experiences from management in Oregon \citep{goheen_eradication_2010}. Over \SI{150}{\hectare} were treated with funding from a number of collaborators but, despite the scale of the treatment efforts, weather conditions conducive to the pathogen and the short cull radius ultimately hampered the effectiveness of control \citep{valachovic_novel_2013}. The disease was discovered inside the national park in two locations in 2014 near Bridge Creek and Bond Creek \citep{stark_sudden_2014}. Management continues with the aim to protect the national park from further infestations.

\subsection{Aims and key questions}

In this chapter we will investigate how to use OCT to design spatial strategies to protect Redwood National Park. Using the open-loop framework previously developed, we will test the strategies on a complex, spatially-explicit model of SOD spread at the landscape scale, and compare the strategies with the \SI{100}{\meter} buffer scheme that was used in practice. We seek to answer two main questions:
\begin{enumerate}
    \item How can OCT be used to design optimal spatial strategies to protect a high value region?
    \item For the Redwood Creek case study, how do the strategies identified using OCT compare with the management that was carried out in practice?
\end{enumerate}

We start by describing the simulation model used in this chapter, and which approximate models are appropriate for spatial optimisation. Using reduced resolution ODE models of the SOD invasion, we optimise objectives to protect the national park whilst also varying the relative benefit of protecting hosts outside the park. We compare the strategies and their wider effects on the surrounding regions. This chapter can be considered an extension of the simple two-patch model in Chapter~\ref{ch:three_patch} to continuous landscapes, applied to the Redwood Creek infestation.

\section{Simulation model}

In Chapter~\ref{ch:intro} (Section~\ref{sec:ch1:sod_models}, p.~\pageref{sec:ch1:sod_models}) we reviewed models of SOD spread. Two models captured SOD dynamics at the landscape scale, the scale appropriate for modelling the invasion into Redwood National Park. The first model by \citet{meentemeyer_epidemiological_2011} was fitted to data and shown to capture spread across California, and has subsequently been used to assess potential large scale management efforts \citep{cunniffe_modelling_2016}. The second model by \citet{tonini_modeling_2018} is similar in structure, but as it is integrated with the forest simulation model LANDIS-II \citep{scheller_design_2007}, the scope for complex control strategies of the form we consider is limited. We therefore use a reimplementation of the model from \citet{meentemeyer_epidemiological_2011} in this chapter. The controls implemented by \citet{cunniffe_modelling_2016} include area-wide removal of susceptible and infected hosts, and test a number of prioritisation strategies, but the controls are only implemented across the entire state. The controls at this scale are ineffective, or at least require infeasible levels of host removal, but the model was not used to investigate whether local controls could be used to protect local regions. Here we will use this model on a smaller scale, modelling the infestation near Redwood National Park.

\subsection{Model structure}

The model is a stochastic, spatially-explicit, raster-based simulation, with each simulation cell containing a number of host units that each represent hosts susceptible to SOD. When conditions are conducive to pathogen sporulation, infected hosts produce inoculum---\emph{P. ramorum} spores---that are distributed according to a dispersal kernel and can infect susceptible host units across a heterogeneous landscape. Conduciveness depends on the host type, temperature and moisture conditions, and time of year. In the form described by \citet{meentemeyer_epidemiological_2011}, the model does not include host demography, and individual species of host are not tracked separately but are instead combined into one amalgamated class of host. Host units can be susceptible or infected (Figure~\ref{fig:ch7:sim_model}(a)). As described in \citet{meentemeyer_epidemiological_2011} an infected cell $i$ infects cell $j$ at time $t$ at rate:
\begin{equation}
    \label{eqn:ch7:inf_rate}
    \psi_{ijt} = \beta\left(\chi_t(f_i)m_{it}c_{it}I_{it}\right)\left(\chi_t(f_j)m_{jt}c_{jt}S_{jt}/N_{\text{max}}\right)K_{ji}
\end{equation}
where:
\begin{itemize}
    \item $\beta$ is the overall maximum spore production rate from each infected host unit;
    \item $\chi_t(f_i)$ is a seasonal indicator variable for hosts of type $f_i$, either 0 or 1 dependent on whether hosts in cell $i$ are able to infect and be infected at time $t$;
    \item $m_{it}$ and $c_{it}$ measure how conducive the weather conditions are for pathogen sporulation (moisture and temperature respectively) for cell $i$ at time $t$, each between 0 and 1;
    \item $I_{it}$ and $S_{it}$ are the numbers of infected and susceptible host units in cell $i$ at time $t$;
    \item $N_{\text{max}}$ is the maximum number of host units in any cell; and
    \item $K_{ji}$ is the dispersal kernel giving the probability of a spore travelling from cell $i$ to cell $j$.
\end{itemize}
The dynamics are simulated using the Gillespie direct method \citep{gillespie_exact_1977}, giving stochastic simulations. Note that in the original implementation a discrete time approximation was made for computational ease, an approximation we do not make here because of the smaller study region. All parameter and variable meanings and default values are given in Appendix~\ref{app:redwood_params}.

\begin{figure}
    \begin{center}
        \begin{overpic}{Graphics/Ch7/SimModel}
            \put(7,77){\includegraphics{Graphics/Ch7/Model}}
         \end{overpic}
        \caption[Simulation model of SOD invading Redwood National Park]{The simulation model capturing the spread of SOD into Redwood National Park. The hosts transition between susceptible and infected, as shown in \textbf{(a)}. Infected hosts produce inoculum which is then distributed according to the kernel shown in \textbf{(b)}, a combination of a long- and short-ranged Cauchy kernel. In \textbf{(c)} the host landscape around the south of the national park is shown, with the red cross indicating the initial infection site and the black line marking the edge of the national park (NP). \textbf{(d)}--\textbf{(g)} plot the median level of infection across 100 realisations of the simulation model over \SI{30}{\years}. The colour indicates the proportion of hosts infected, and the transparency indicates the host density. \textbf{(h)} and \textbf{(i)} show the disease progress curves across the landscape and just in the national park respectively. The deciles of the distributions and the median are shown in each case.\label{fig:ch7:sim_model}}
    \end{center}
\end{figure}

We use a spatially restricted subset of the same host landscape used by \citet{meentemeyer_epidemiological_2011}, in which each \SI{250 x 250}{\meter} cell is given a localised host index value that combines abundance and susceptibility across all hosts in the cell. This index is then discretised to give a number of host units per cell, with a maximum of 100 in any cell (Figure~\ref{fig:ch7:sim_model}(c)). Since the hosts are conducive to pathogen sporulation at different times of the year, each cell is classified as predominantly redwood and tanoak, or mixed evergreen forest. This forest type mask sets the seasonal indicator variable for each cell ($\chi_t(f_i)$). Redwood/tanoak forest can infect and is able to be infected for the first 28 weeks of the year. Mixed evergreen forests are suitable for the pathogen from week 6 through to week 28.

Model fits by \citet{meentemeyer_epidemiological_2011} indicate infected host units produce spores at a rate of \SI{4.55}{\per\week} \citep{cunniffe_modelling_2016}, which are distributed in space according to the dispersal kernel found by \citet{meentemeyer_epidemiological_2011}. The kernel is a combination of 2 Cauchy type kernels, giving the probability of a spore dispersing a distance $d$:
\begin{equation}
    K(d) = \gamma\left(\frac{N_1}{1+(d/\alpha_1)^2}\right) + (1-\gamma)\left(\frac{N_2}{1+(d/\alpha_2)^2}\right)
\end{equation}
where $\gamma=0.9947$ is the proportion of dispersal events distributed according to the short range kernel, $\alpha_1=$~\SI{20.57}{\meter} is the short range kernel scale, and $\alpha_1=$~\SI{9.504}{\km} is the long range kernel scale (Figure~\ref{fig:ch7:sim_model}(b)). The constants $N_1$ and $N_2$ are normalising constants for the two kernels in the expression reported here. Infection is seeded in the cell corresponding to Cookson Ranch, where the initial infestation was found, with that cell starting fully infected. Overall, simulating the model around the southern tip of Redwood National Park leads to widespread infection over \SI{30}{\years} (Figure~\ref{fig:ch7:sim_model}(d)--(g)). The pathogen tends to spread more to the west where conditions are more conducive to pathogen spread (Figure~\ref{fig:ch7:weather}(a)--(b)).

\subsection{Sporulation conditions}

The simulation model used by \citet{meentemeyer_epidemiological_2011} uses the forest type mask that varies pathogen suitability in space and time, and weather data ($m_{it}$ and $c_{it}$ in Equation~\ref{eqn:ch7:inf_rate}) that also varies by cell and each week. This gives rise to the increased spread to the west seen in Figure~\ref{fig:ch7:sim_model}(d)--(g). Since we will be approximating the simulation model using an ODE system, removing this time dependence would make the ODEs simpler and make convergence using OCT easier. It would also improve the computational efficiency of running the simulation model. We tested what effect averaging these time-dependent variables has on the simulations, generating a single time-independent scaling factor for each cell that captures the effects of temperature, moisture and forest type (Figure~\ref{fig:ch7:weather}).

To calculate the average, we take for each cell $i$ the root mean square value of the weather and forest type mask over time:
\begin{equation}
    M_i = \sqrt{\frac{\sum_t\left(\chi_t(f_i)m_{it}c_{it}\right)^2}{N_t}}
\end{equation}
where $N_t$ is the number of time points. The root mean square is used because in the infection rate (Equation~\ref{eqn:ch7:inf_rate}) these terms appear in the susceptible and infected terms, and so are effectively squared. Taking a simple mean leads to significantly lower levels of infection. Since the kernel is very short ranged, taking this local average is a good approximation, since in reality the value is not squared but multiplied by the value in another cell. To understand why this is necessary, consider a single cell with conduciveness equal to 0.5 for half of each year, and zero otherwise. Whilst using the mean conduciveness of 0.25 leads to a relative infection rate of 0.0625 ($0.25^2$), the actual mean relative infection rate is 0.125 ($0.5^2$ for half the year, and zero otherwise). The root mean square gives the correct relative infection rate.

It is clear that the redwood/tanoak forest nearer the coast is generally more suitable for the pathogen, because of both the weather and forest type mask (Figure~\ref{fig:ch7:weather}(a)--(b)). We ran 250 simulations over \SI{18}{\years} using the full weekly weather and forest type, averaging just the weather, and averaging both the weather and forest type over time. Whilst averaging these effects over time does reduce within year variation, it does not have a large effect on the median simulation dynamics (Figure~\ref{fig:ch7:weather}(c)). We therefore use the averaged weather and forest type mask for simulations in this chapter.

\begin{figure}
    \begin{center}
        \includegraphics{Graphics/Ch7/Weather}
        \caption[Acounting for weather and forest type in the simulation model]{Accounting for varying weather and forest type in the simulation model. In \textbf{(a)} the forest type around the national park is shown, with regions of mixed evergreen forest, and redwood and tanoak areas. The redwood/tanoak forest is susceptible and can sporulate for the first 28 weeks of the year. Mixed evergreen forests are unsuitable for the pathogen until the 7\textsuperscript{th} week of the year, and become unsuitable again after the 28\textsuperscript{th} week. \textbf{(b)} shows the average mask across the region, with weather and forest type averaged over time. The weather and forest type are more suitable for the pathogen closer to the coast. In \textbf{(c)} the effect of averaging the weather and forest type mask is shown. The lines show the medians of 250 simulation realisations. The full simulation has weekly varying weather, and differences in pathogen suitability due to forest type. Averaging just the weather over time, or averaging both the weather and forest type suitability over time, does not significantly affect the simulation dynamics. We therefore use the averaged weather and forest type mask for simulations going forward, because of the increased speed of simulation.\label{fig:ch7:weather}}
    \end{center}
\end{figure}

\section{Approximate model}

\subsection{Reduced resolution model}

We require an approximate model of the simulation dynamics which must be simple enough to allow optimisation using OCT, but with enough spatial detail to allow spatially resolved strategies to be identified. As mentioned in Chapter~\ref{ch:complex_models}, the main factor affecting convergence in OCT is the number of variables that must be optimised. As spatial detail is added this increases the number of state and control variables, but also increases the spatial resolution of the control strategy. We therefore seek an approximate model that captures as much spatial detail as possible whilst remaining optimisable. We propose using reduced resolution ODE models of the simulation, which are raster-based ODE approximations of the dynamics on a grid of a larger spatial scale (Figure~\ref{fig:ch7:approx_model}(a)--(b)). The host landscape used is an aggregated version of the simulation host landscape, with each cell in the approximate model using the average host density of the \SI{250}{\meter} simulation cells contained within it. A buffer region is included in the simulation model around the approximate landscape (Figure~\ref{fig:ch7:approx_model}(a)). This ensures that any edge effects from infection leaking around the managed region are accounted for. This more realistically models the effect of inoculum pressure from outside the area under management.

The ODE approximation follows the same susceptible-infected dynamics as the simulation model, with dispersal across the landscape according to a different kernel. Since the approximate model uses grid cells of a different scale to the simulation and is deterministic, the same kernel cannot be lifted from the simulation. We use the averaged weather and forest type mask to modulate the susceptibility and infectiousness of each cell, with the average now calculated for each cell in the aggregated landscape. The ODE system for susceptible ($\tilde{S}$) and infected ($\tilde{I}$) hosts in aggregated cell $i$ is therefore given by:
\begin{subequations}
    \label{eqn:ch7:approx_model}
    \begin{align}
        \dot{\tilde{S}}_i &= -\tilde{\beta}M_i\tilde{S}_i\sum_j\left(k_{ij}M_j\tilde{I}_j\right) \\
        \dot{\tilde{I}}_i &= \tilde{\beta}M_i\tilde{S}_i\sum_j\left(k_{ij}M_j\tilde{I}_j\right)
    \end{align}
\end{subequations}
where $\tilde{\beta}$ is the infection rate to be fitted, $k_{ij}$ is the kernel also to be fitted, $M_i$ is the averaged weather and forest type mask, and the sum is over all aggregated cells in the landscape.

\subsection{Metrics for comparison}

To choose the most appropriate resolution for the approximate model, by testing the quality of fit and plausibility of control optimisation, we require metrics that can compare the approximate model with simulations across different resolutions. This will ensure that we can assess different resolution approximate models using a single consistent metric. There are three obvious choices of scale for comparing a single approximate model with simulations:
\begin{description}
    \setlength{\itemsep}{3pt}%
    \setlength{\parskip}{3pt}%
    \setlength{\parsep}{3pt}%
    \item[Landscape scale:] total infection across the landscape;
    \item[Aggregated scale:] simulation data aggregated to the approximate model resolution;
    \item[Divided scale:] approximate model data resampled down to the \SI{250}{\meter} simulation scale.
\end{description}

For clarity, in a simulation the number of infected host units in cell $i$ is given by $I_i(t)$. This is the divided scale. At the landscape scale the numbers are simply summed over all cells ($\sum_iI_i(t)$). At the aggregated scale the numbers are summed over the set of cells contained within the aggregated cell $j$ ($A_j$):
\begin{equation}
    \tilde{I}_j = \sum_{i\in{}A_j}I_i
\end{equation}
To calculate the number of hosts at the divided scale from the approximate model, the hosts are homogeneously distributed across simulation cells contained within the aggregated cell:
\begin{equation}
    I_i = \frac{\tilde{I}_j}{N_j} \quad \forall i \in A_j
\end{equation}
where $N_j$ is the number of simulation cells contained within aggregated cell $j$.

For comparing across different resolutions, the aggregated scale cannot be used since this will be a different absolute scale for each approximate model. The landscape and divided scales can be used to compare across resolutions, though. The metric we choose for comparison is the root mean square error (RMSE) between the simulation and approximate model disease progress curves (DPCs). This metric is a scaled version of the summed squared errors (SSE) used in previous chapters. We use the RMSE in preference to the SSE because the value is more meaningful for comparison, since it is measured in the same units as disease progression. At the landscape scale, the RMSE for the landscape DPCs is used. At the divided and aggregated scales the RMSE is calculated over DPCs for each cell over \SI{30}{\years} with a time step of 2~weeks.

\subsection{Fitting}

We fit two forms of kernel ($k_{ij}$ in Equation~\ref{eqn:ch7:approx_model}), an exponential type:
\begin{align}
    k_{ij}^1 &\propto \exp{\left(-d_{ij}/\sigma_1\right)}\;, \\
\intertext{and a Cauchy type kernel:}
    k_{ij}^2 &\propto \frac{1}{1+\left(d_{ij}/\sigma_2\right)^2}\;,
\end{align}
where $d_{ij}$ is the distance between cell centres for cells $i$ and $j$, and $\sigma_1$ and $\sigma_2$ are scale parameters for the two forms of kernel. Whilst other forms of kernel could be considered---for example an exponential power distribution as considered by \citet{skelsey_pest_2013} that can capture both short- and fat-tailed shapes---there was little difference in fit between the exponential and Cauchy kernel models. We therefore only consider these possibilities. The scale parameters and infection rate ($\tilde{\beta}$) are fitted by minimising the RMSE at the aggregated scale using the Nelder-Mead simplex algorithm \citep{gao_implementing_2012} from the SciPy library \citep{scipy}. This scale is used to maintain spatial information whilst avoiding repeated resampling operations that would be required to calculate the metric at the divided scale.

The fitted dynamics match the simulation very closely at the landscape scale (Figure~\ref{fig:ch7:approx_model}(c)), with the largest errors spatially close to the infection introduction site (Figure~\ref{fig:ch7:approx_model}(d)--(f)). In spatial pattern, all reduced resolution models capture the increased spread to the west where weather and forest type conditions are more suitable for the pathogen (Figure~\ref{fig:ch7:resolution_snapshots}). Fitted parameters for each approximate model are given in Appendix~\ref{app:redwood_fits}.

\begin{figure}[H]
    \begin{center}
        \includegraphics{Graphics/Ch7/ApproxModel}
        \caption[Approximating the simulation model with a reduced resolution model]{Approximating the simulation model with a reduced resolution model. The simulation host landscape in \textbf{(a)} includes a buffer to eliminate edge effects. The red box shows the region approximated by the reduced resolution host landscape, shown using a \SI{2500}{\meter} resolution in \textbf{(b)}. The ODE approximate model is fitted to the simulation model by matching DPCs for each cell in the approximate model, here shown for the \SI{2500}{\meter} resolution model with a Cauchy kernel. In \textbf{(c)} the simulation and approximate disease progress curves across the whole landscape are shown, with the approximate model capturing the dynamics well. In \textbf{(d)} the root mean square error (RMSE) for each cell at the scale of the simulation is shown (divided metric). The largest errors are seen close to the initial seed infection. \textbf{(e)} shows the disease progress curves for a single \SI{250}{\meter} cell far from the initial seed infection, capturing the dynamics well. \textbf{(f)} shows the same for a cell close to the initial infection. The low resolution approximate model cannot precisely capture the number of hosts in each \SI{250}{\meter} cell, and so the dynamics are captured less well.\label{fig:ch7:approx_model}}
    \end{center}
\end{figure}

\begin{figure}
    \begin{center}
        \includegraphics{Graphics/Ch7/ResTestingSnapshots}
        \caption[Comparing disease spread across approximate model resolutions]{Comparing disease spread across approximate model resolutions. The top row shows infection spread in the \SI{2500}{\meter} resolution approximate model after 10, 20, and \SI{30}{\years}. The second and third rows show the same for the \SI{1500}{\meter} and \SI{250}{\meter} resolution models respectively. All model resolutions capture the same pattern of spread, with infection first spreading towards the coast.\label{fig:ch7:resolution_snapshots}}
    \end{center}
\end{figure}

\section{Control and optimisation\label{sec:ch7:optim_methods}}

\subsection{Control methods}

In the simulation and reduced resolution approximate models, we implement control methods that remove infectious or susceptible hosts at a continuous rate. Optimising control finds a partition between removing susceptible hosts and infected hosts across the landscape, in order to achieve an objective. The roguing control method removes infectious hosts, and the thinning method removes healthy, susceptible hosts. The time-dependent variables $u_i(t)$ and $v_i(t)$, each between 0 and 1, give the level of thinning and roguing respectively in cell $i$ of the approximate model. With $\eta$ as the rate of removal, the approximate model dynamics become:
\begin{subequations}
    \begin{align}
        \dot{\tilde{S}}_i &= -\tilde{\beta}M_i\tilde{S}_i\sum_j\left(k_{ij}M_j\tilde{I}_j\right) - u_i(t)\eta{}\tilde{S}_i\\
        \dot{\tilde{I}}_i &= \tilde{\beta}M_i\tilde{S}_i\sum_j\left(k_{ij}M_j\tilde{I}_j\right) - v_i(t)\eta{}\tilde{I}_i\;.
    \end{align}
\end{subequations}

We impose a simple constraint on the thinning and roguing levels:
\begin{equation}
    \sum_i\left(u_i(t) + v_i(t)\right) \leq 1 \quad \forall t\;,
\end{equation}
where the sum is over all cells in the approximate model. This means the total control rate $\eta$ is partitioned into thinning and roguing, and across all cells in the approximate model. This captures the logistical constraint of limited resources to allocate to particular locations in the landscape. Note that the constraint here is on intensity of control, not the number of hosts removed. This captures the difficulty of resource location; control resources allocated to cells with fewer hosts to remove still have to be moved to that location. This constraint is simpler in form than those used in previous chapters. By separating the state from the control constraint, convergence when optimising the control is significantly improved. Controls are lifted from the approximate model to the simulation model as expected, with control homogeneously applied across simulation cells contained within each approximate model cell. As before, disease is seeded at Cookson ranch, and control starts after \SI{3}{\years} of uncontrolled spread to allow the infestation to establish.

\subsection{Objective function}

We use an objective function that maximises the number of susceptible hosts at the final time $T=\text{\SI{30}{\years}}$, with a weight for each cell in the approximate model landscape. The weight for hosts inside the national park is equal to 1. For cells that are partially outside the park, this weight is multiplied by the proportion inside the national park. We test different objectives, varying the weight of hosts outside the national park. The overall objective is given by:
\begin{equation}
    J = f_1\sum_ip_iS_i(T) + f_2\sum_i(1-p_i)S_i(T)
\end{equation}
where $f_1$ and $f_2$ are the weights for hosts inside and outside the national park respectively, and $p_i$ is the proportion of cell $i$ that is inside the national park.

We test three different objectives (Figure~\ref{fig:ch7:objectives}). The Total objective uses $f_2=1$ so that all hosts across the landscape are weighted equally. The NP objective only considers the national park, with $f_2=0$. Finally, the Mixed objective uses $f_2=0.5$, as a balance between the first two objectives.

\begin{figure}
    \begin{center}
        \includegraphics{Graphics/Ch7/Objectives}
        \caption[Objective functions under test]{The different objective functions under test. \textbf{(a)}, \textbf{(b)} and \textbf{(c)} show the Total, NP and Mixed objective functions respectively.\label{fig:ch7:objectives}}
    \end{center}
\end{figure}

\subsection{Large scale optimisation}

Despite reducing the resolution of the approximate model, the optimisation problem is still very large. For a \SI{2.5}{\km} resolution approximate model, there are 120 cells each with two state variables and two control variables. Discretising the \SI{30}{\year} time period into 120 steps, each step representing 7 weeks as infection is only possible for 28 weeks of the year, gives approximately \num{58000} variables to optimise using the direct method. The software used in previous chapters to carry out the direct optimisation, BOCOP \citep{bocop}, is unable to handle problems of this size. For the optimisations in this chapter we use the optimiser underlying BOCOP directly, Ipopt \citep{wachter_implementation_2006}, and generate the non-linear programming (NLP) problem ourselves.

The calculation of the NLP jacobian, i.e.\ the derivative of the NLP constraints with respect to the NLP variables, is the limiting factor in using BOCOP for large scale optimisation, since BOCOP uses automatic numerical differentiation to calculate this derivative. Since we are able to compute the equations and hence the derivative exactly, we implement exact derivatives in our interface to Ipopt. This allows for faster convergence and lower memory usage. We describe the formulation of the NLP problem in Appendix~\ref{app:redwood_form} and the optimisation details in Appendix~\ref{app:redwood_opt}. Despite these improvements in optimisation, the highest resolution for which optimisation is successful is the \SI{2.5}{\km} resolution approximate model.

\subsection{Buffer strategy}

We compare the optimal spatial strategies with the control that was carried out in practice. Once the source of infection was located along Redwood Creek, a \SI{100}{\meter} radius buffer control was implemented. Any infected hosts that are detected are removed along with all susceptible hosts within \SI{100}{\meter}. To implement this strategy using our model, within each cell all host units are assigned a random position. At the end of each year, surveillance is carried out and infected hosts are detected according to a Bernoulli trial with probability of success $p=0.7$, following the same method used by \citet{cunniffe_modelling_2016}. These detected hosts and any hosts within \SI{100}{\meter} are then removed, following the strategy used in practice.

\section{Results}

\subsection{Resolution testing}

We first describe the results of resolution testing: assessing different resolution approximate models. We use the RMSE metric at both the landscape and divided scales to compare across resolutions. Under no control, resolutions from \SI{250}{\meter} to \SI{2.5}{\km} all perform similarly at the landscape scale, with the Cauchy kernel giving slightly lower metric values (Figure~\ref{fig:ch7:resolution_testing}(a)). At the divided scale higher resolution models perform better, as predictions are more accurate since host density information is available at a finer spatial scale. RMSE values are much lower at the landscape scale because errors in host numbers in any individual cell are averaged out across the landscape.

To ensure the approximate models fit when control strategies are implemented, we also test the approximate models under an optimised non-spatial control strategy. A non-spatial strategy is used so that exactly the same control can be used across all resolution models. The strategy is optimised as described in Section~\ref{sec:ch7:optim_methods} using the \SI{2.5}{\km} resolution model and the Total objective function, with the added constraint that control is allocated evenly across the landscape. The resulting control carries out thinning first, before switching to roguing (Figure~\ref{fig:ch7:resolution_testing}(c)). Under this control strategy all resolutions fit closely to simulations realised under the same strategy (Figure~\ref{fig:ch7:resolution_testing}(b)). High and low resolution models match the simulation disease progress curves well and once again the largest errors are seen close to the introduction site (Figure~\ref{fig:ch7:resolution_testing}(d)--(e)).

We choose to use the \SI{2.5}{\km} resolution approximate model for the optimisations in the rest of this chapter. This resolution provides enough spatial detail to capture the increased spread to the west and allow spatially resolved control strategies, and also fits well to simulation data. Whilst higher resolution models provide improved fits, for resolutions higher than \SI{2.5}{\km} the optimal control problem is not tractable using our methods. Later in this chapter we will compare the resulting optimal control using this approximate model with that from a lower, \SI{5}{\km} resolution model, and show that control is improved by the additional spatial detail in the \SI{2.5}{\km} model.

\begin{figure}
    \begin{center}
        \includegraphics{Graphics/Ch7/ResTesting}
        \caption[Using the RMSE metric to test the approximate models]{Using the RMSE metric the different resolution approximate models can be tested. In \textbf{(a)} the landscape and divided metrics are shown for the exponential and Cauchy kernel under no control. The Cauchy kernels show a slightly better fit across all resolutions, with higher resolution approximate models fitting better. Under a non-spatial control strategy, all resolutions fit approximately the same and, since the level of infection is lower, fit better than under no control, as shown in \textbf{(b)}. The control strategy is shown in \textbf{(c)}. The landscape disease progress curves are shown in \textbf{(d)}, and the divided RMSE metric across the landscape for the \SI{2500}{\meter} Cauchy model under control is shown in \textbf{(e)}.\label{fig:ch7:resolution_testing}}
    \end{center}
\end{figure}

\newpage
\subsection{Optimal spatial control}

The control strategies optimised using the \SI{2.5}{\km} resolution approximate model are applied to the simulation model using the open-loop framework described in previous chapters. Using the NP objective, the optimal strategy initially rogues at the disease introduction site alongside heavy thinning in the highly pathogen conducive region to the west (Figure~\ref{fig:ch7:ol_np}(d)--(e)). Thinning is then carried out more generally outside the national park, with intensive roguing inside the park. The disease progress curves across the landscape and inside the national parks (Figure~\ref{fig:ch7:ol_np}(b) and (c)) show low levels of infection, and the approximate model matches the average landscape dynamics. Median simulation dynamics show significant loss of host outside the national park, but the park itself is well protected (Figure~\ref{fig:ch7:ol_np}(f)). Note though, that significant amounts of infection spread around the edge in the buffer region where management is not carried out.

The NP objective protects Redwood National Park, but at a significant cost to the surrounding region. An alternative strategy, the Mixed objective, weights protection of hosts outside the national park by half as much as those inside (Figure~\ref{fig:ch7:ol_mixed}(a)). The optimal strategy under this objective still thins to the west, but uses roguing rather than thinning in the east (Figure~\ref{fig:ch7:ol_np}(d)--(e)). This results in higher levels of infection (Figure~\ref{fig:ch7:ol_np}(b)--(c)), but retains more healthy host in the north east of the region. As with the NP objective, infection does spread around the buffer region.

We next analyse the performance of the control strategy that was carried out in practice, the \SI{100}{\meter} buffer. The buffer strategy keeps levels of infection much lower than the previous strategies, because these hosts are directly detected and removed every year. In some cases the buffer strategy successfully manages the epidemic, keeping levels of infection very low and with few hosts removed (Figure~\ref{fig:ch7:100m_buffer}(b)--(c)). In many cases though, the buffer of \SI{100}{\meter} is not sufficient to control the spread and, whilst overall infection levels remain low, a large number of hosts are eventually removed in the national park. This means that the median simulation dynamics show large numbers of hosts removed across the landscape, but the strategy does still protect many hosts in the north east of the region.
{
\makeatletter
\setlength{\@fptop}{10\p@} 
\setlength{\@fpbot}{0\p@ \@plus 1fil}
\makeatletter

\begin{figure}[H]
    \begin{center}
        \makebox[\textwidth][c]{\includegraphics{Graphics/Ch7/OL_NP}}
        \caption[Optimal control strategy to protect the national park]{Optimal control strategy using the NP objective. \textbf{(a)} shows the objective raster, with control optimised to maximise the number of healthy hosts in the national park after \SI{30}{\years}. The simulation and approximate model disease progress curves across the landscape are shown in \textbf{(b)}, and the proportion of hosts infected or removed within the national park only in \textbf{(c)}. These DPCs do not include the buffer region in the simulation model. The control strategy is shown in \textbf{(d)} and \textbf{(e)}, with thinning initially focussed near the coast on the west, and roguing focussed in the national park. \textbf{(f)} The median proportion of hosts infected in the simulations show a reduction in hosts across the landscape apart from in the national park. The colour indicates the proportion of hosts infected, and transparency indicates the host density. There is little infection spread inside the managed area, but infection does spread around the edge in the buffer region.\label{fig:ch7:ol_np}}
    \end{center}
\end{figure}

\begin{figure}[H]
    \begin{center}
        \makebox[\textwidth][c]{\includegraphics{Graphics/Ch7/OL_Mixed}}
        \caption[Optimal control strategy using the Mixed objective]{\textbf{(a)} The Mixed objective protects the national park, but with some value given to hosts outside. The disease progress curves across the landscape are shown in \textbf{(b)}, and the proportion of hosts infected or removed within the national park only in \textbf{(c)}. There is more disease than using the NP objective. The control strategy shown in \textbf{(d)} and \textbf{(e)} only thins in the environment conducive to sporulation. More roguing than thinning is carried out to the east, leading to more infection but also more host retained (\textbf{(f)}).\label{fig:ch7:ol_mixed}}
    \end{center}
\end{figure}
}

\begin{figure}
    \begin{center}
        \makebox[\textwidth][c]{
        \begin{overpic}{Graphics/Ch7/100mBuffer}
            \put(2.5,38){\includegraphics{Graphics/Ch7/Model2}}
         \end{overpic}
        }
        \caption[\SI{100}{\meter} buffer zone strategy]{To implement the \SI{100}{\meter} buffer strategy, a detected class is included, as shown in \textbf{(a)}. Once per year infected hosts are detected by a Bernoulli trial with probability of success $p=0.7$. The \SI{100}{\meter} buffer removes detected hosts, and all hosts within \SI{100}{\meter} of the detection once every year, as illustrated by the red circular buffer. The strategy keeps levels of infection low (\textbf{(b)}) as it directly removes these hosts, but can lead to high numbers of hosts removed in the national park (\textbf{(c)}). As shown in \textbf{(d)}, large numbers of hosts are removed across the landscape.\label{fig:ch7:100m_buffer}}
    \end{center}
\end{figure}

\FloatBarrier
\subsection{Comparing strategies}

The different control strategies vary in the pattern of hosts saved across the landscape (Figure~\ref{fig:ch7:strat_comparison}(a)). The NP and Mixed objectives very effectively protect the national park, whereas without control spatially optimised to protect it, the non-spatial and \SI{100}{\meter} buffer strategies cannot protect the national park effectively. In the cases where it fails to control the spread, the \SI{100}{\meter} buffer strategy also results in more hosts culled than any of the other strategies (Figure~\ref{fig:ch7:strat_comparison}(b)). Whilst this strategy does not remove many hosts early in the epidemic, if disease persists then very large areas of host may need to be controlled. Under the OCT strategies however, more hosts are removed pre-emptively, but this does result in improved disease management.

\begin{figure}
    \begin{center}
        \makebox[\textwidth][c]{\includegraphics{Graphics/Ch7/StrategyComparison}}
        \caption[Comparison of all control strategies]{Comparison of all control strategies. In \textbf{(a)} the proportion of hosts saved relative to simulations under no control are shown for each strategy, with the national park outlined in black. \textbf{(b)} shows the area of host culled for each strategy, and \textbf{(c)} shows the levels of thinning and roguing for each of the OCT strategies. \textbf{(d)} and \textbf{(e)} give the area of host saved inside and outside the national park respectively. Horizontal lines indicate the median, and vertical lines span the 25\textsuperscript{th} to 75\textsuperscript{th} percentiles. In some cases the \SI{100}{\meter} buffer strategy contains the epidemic and saves the most hosts, but can also result in culling large numbers of hosts. The NP and Mixed strategies protect the national park most effectively.\label{fig:ch7:strat_comparison}}
    \end{center}
\end{figure}

The OCT strategies (Total, NP, Mixed and Non-Spatial) all show a similar time dependence in allocation to thinning or roguing controls (Figure~\ref{fig:ch7:strat_comparison}(c)), but with dependence on spatial location. Initially control is split between both methods. Thinning dominates the start of the epidemic, before switching to roguing after approximately \SI{10}{\years}. On average, the highest number of hosts outside the national park are protected by the Mixed strategy (Figure~\ref{fig:ch7:strat_comparison}(d)). Hosts within the national park are best protected by the NP strategy, with the Mixed strategy finding a good balance of protection both in and out of the park.

\subsection{Resolution choice}

We here test how appropriate our choice of the \SI{2.5}{\km} resolution approximate model was. The motivation for this choice was that this model is the highest resolution for which the NLP optimisation is tractable. Higher spatial resolution allows more spatially detailed strategies, and hence improved protection of the national park. Figure~\ref{fig:ch7:control_res_compare} compares control performance of the \SI{2.5}{\km} model with a \SI{5}{\km} resolution model. Control in the lower resolution model is less spatially resolved around the national park. This leads to fewer hosts saved, both inside and outside the national park. It is clear that for effective control a higher resolution model is preferable, and so within our methodological limitations the \SI{2.5}{\km} resolution approximate model is best.

\begin{figure}[H]
    \begin{center}
        \makebox[\textwidth][c]{\includegraphics{Graphics/Ch7/ControlResComparison}}
        \caption[Comparison of optimal control using the \SI{2.5}{\km} and \SI{5}{\km} models]{Comparison of optimal control using the \SI{2.5}{\km} and \SI{5}{\km} approximate models. \textbf{(a)} shows the proportion of hosts saved by each OCT control strategy using the \SI{2.5}{\km} resolution approximate model (i.e.\ repeats part of Figure~\ref{fig:ch7:strat_comparison}(a)), with the same for the \SI{5}{\km} model shown in \textbf{(b)}. \textbf{(c)} and \textbf{(d)} show the area of host saved inside and outside the national park respectively, for each model and for each spatial OCT strategy. The higher resolution \SI{2.5}{\km} model saves more hosts on average because the control can be more spatially detailed, and hence it protects the national park better.\label{fig:ch7:control_res_compare}}
    \end{center}
\end{figure}

\newpage
\section{Discussion}

In this chapter we have shown how spatially optimised strategies can protect valuable regions from disease. Using OCT to inform spatial control improves disease management, and better protects Redwood National Park from SOD than the control that was actually applied. The optimal strategies make use of epidemiological features which were ignored by the management carried out in practice. In particular, the optimal spatial strategies focus on thinning controls in areas where weather conditions are most conducive to pathogen spread.

\subsection{Spatial optimisation}

To allow for spatial optimisation of control, we used a reduced resolution deterministic model approximating the dynamics of the fully spatially-explicit simulation model. Capturing the full spatial detail of the simulation model results in a model for which optimisation is not tractable. In a similar setting, work by \citet{epanchin_optimal_2012} optimised spatial strategies for control of invasive species, but with much greater simplifications to the model dynamics. To allow spatial optimisation, \citet{epanchin_optimal_2012} reduce the underlying spread model to an integer problem where cells are either invaded or not, and the species potentially spreads to nearest neighbour cells in each time step. In Chapters \ref{ch:three_patch} and \ref{ch:complex_models} we simplified spatial dynamics by modelling the system with two or three metapopulations. By applying large-scale direct optimisation to the spatial optimisation problem, we have significantly reduced the simplifications that are necessary, and extended the optimal control to a discretised continuous landscape. This means that the resulting strategies are driven more closely by real-world dynamics, and so will be more effective if applied in practice. In testing the strategies on a fully spatially-explicit model---by applying the control lifting framework used in Chapters~\ref{ch:complex_models} and \ref{ch:protect_tanoak_control}---we have demonstrated how these strategies would perform in reality, and shown that they outperform the management that was actually applied.

The constraint we used in this chapter partitioned control resources between thinning and roguing across the landscape. We did not use a constraint limiting the number of hosts controlled, as we used in Chapters \ref{ch:three_patch} and \ref{ch:protect_tanoak_control} for example, because for an NLP problem of this scale these mixed constraints slow convergence. Including a mixed constraint could change the optimal strategy, but the general spatial structure of control, and the partitioning between thinning and roguing is unlikely to be significantly different. The optimal spatial strategies we found here show similar features to the results of previous chapters. We see a switching strategy that switches from thinning in a region that would threaten the national park, to prioritising roguing within the park. This is very similar to the switching strategies found in Chapter~\ref{ch:three_patch}, where a buffer region was first prioritised, before switching to control in the high value region.

Other methods exist for finding an optimal spatial control, including using OCT on PDE models of disease spread rather than the ODE approximate models we have used here. PDE models have been used previously to model plant disease epidemics \citep[e.g.][]{white_role_2006}. Reaction-diffusion type models can be used to capture the spatio-temporal dynamics of disease spread, and OCT can be applied to the resulting model to optimise spatial controls. For example, work by \citet{neilan_optimal_2011} optimised spatial deployment of rabies vaccines to wild raccoons. They also found that optimal strategies make use of landscape features, in their case taking advantage of natural barriers to spread. \citet{miyaoka_optimal_2019} also optimise a PDE model of disease spread, there a model of Zika virus spread and optimal vaccine deployment. Whilst these studies use OCT to find spatial strategies, similar to the optimisations we have carried out, neither study tests the results on a more detailed simulation model. The advantage of our approach is that the effectiveness of the control can be shown under more realistic conditions, and control is designed to protect a high value region. Here we showed that the OCT strategies are still effective at protecting Redwood National Park under the stochastic spread in the simulation model.

\subsection{Spatial resolution}

We optimised spatial controls in this chapter using a \SI{2.5}{\km} resolution approximate model. This key choice of resolution was a balance between increasing spatial detail and a tractable optimisation problem. With the optimiser and NLP problem we use here, a higher resolution model is not solvable. We showed however, that lower resolution models result in less effective optimal spatial controls because of the lack of spatial detail. Whilst higher resolution models would be preferable, and could be used with sufficient computational resources, we here use the highest resolution model possible with our setup.

\subsection{Practical implementation}

Using the simulation model to test the OCT results allows potential practical outcomes of strategies to be compared. The OCT strategies tested here perform best on average, and protect the whole region against the very worst epidemics, where the \SI{100}{\meter} buffer strategy results in removal of very large numbers of hosts. However, the strategies do result in large scale removal of hosts across the landscape. In the reduced resolution approximate model this extensive control is required to manage disease, but in the stochastic simulation---and by extension the real world---this may not be absolutely necessary. Implementing large scale removal can be difficult, particularly if centralised management leads to conflicts with local landowners who may rely on the forest for income \citep{alexander_lessons_2010}. Opposition from affected parties can impact on deployment of control, as was the case with citrus canker \citep{gottwald_citrus_2007}. As shown by the testing of the \SI{100}{\meter} buffer strategy, in some cases the disease does not spread rapidly and considerably lower levels of control are sufficient. 

Deciding on the scale of control to carry out, or when to switch to larger scale control, is a difficult problem, particularly when estimates of future epidemic size can be very uncertain \citep{neri_bayesian_2014}. With citrus canker, choosing this scale incorrectly eventually led to failure of control \citep{gottwald_citrus_2007}. Our models have shown that the more extensive OCT strategies perform better than the \SI{100}{\meter} buffer on average, but there is a trade-off between implementing simpler controls more rapidly, and carrying out the optimal strategy with more knowledge of the system \citep{thompson_control_2018}. Whilst these decisions must be made in collaboration with policy makers and landowners, we have demonstrated that larger scale control is necessary to give the most effective control on average. For comparison, management of the isolated outbreak in Oregon treated approximately \SI{2510}{\hectare} of land between 2001 and 2015, costing \$5000--15000 per hectare \citep{goheen_sudden_2017}. The total area of the approximate model we consider here is \SI{75000}{\hectare}, so the level of control necessary is extensive.

We have not accounted in our model for the importance of Redwood Creek. \emph{P.~ramorum} can spread through the watercourse, and this could lead to increased rates of spread into the national park. It could also allow for more effective surveillance and control strategies that prioritise areas along the river. Our work in this chapter has shown how the pathogen suitability of the landscape can influence optimal strategies, here through prioritising thinning where the weather is most conducive to pathogen spread. Similar methods could improve control when additional spread through the river is taken into account, but additional data would be necessary to fully inform this type of strategy. There is currently little information about the rate of \emph{P.~ramorum} spread through watercourses.

The broader message of the OCT strategies here though, is that management at larger scales is necessary to be sure of effective disease control. The \SI{100}{\meter} buffer was clearly too short to protect Redwood National Park. This is consistent with work by \citet{cunniffe_modelling_2016} that found the optimal radius of treatment to be \SI{187.5}{\meter}, but their result does depend on how regions are prioritised for treatment and the risk aversion of the decision maker. We also see in our results the benefit of front-loading identified by \citet{cunniffe_modelling_2016}, i.e.\ allocating additional control resources earlier in the epidemic. Thinning resources in all OCT strategies are carried out early during management, in order to slow the spread of infection in the most susceptible regions. The control is carried out ahead of the epidemic wavefront in anticipation of future spread. The \SI{100}{\meter} buffer strategy, on the other hand, is purely reactive to the ongoing epidemic, and as a result must try to catch up with the epidemic wavefront. The benefit of this concept of front-loading has also been seen in other optimal control theory studies \citep{behncke_optimal_2000}.

\subsection{MPC}

A further extension to this work would be to include the feedback framework of MPC developed in Chapter~\ref{ch:complex_models} and used in Chapter~\ref{ch:protect_tanoak_control}. This could help to tailor the OCT strategies to disease progression, reducing levels of control when the epidemic is slow to spread. As we discussed previously, the \SI{100}{\meter} buffer strategy shows that lower levels of control can be sufficient. The MPC framework could help adapt to an individual epidemic, to increase or decrease control intensity as required. Similar to the results in Chapter~\ref{ch:complex_models}, this could improve control under the stochasticity of the simulation model. Allowing parameter inference between MPC update times could further improve control by updating the approximate model as the epidemic proceeds \citep{thompson_control_2018}.

\section{Conclusions}

In conclusion, this chapter has demonstrated the extension of the optimal control methods we have developed to optimising spatial strategies. These optimal spatial strategies can protect high value regions threatened by disease more effectively. The OCT strategies perform better on average than the control that was actually carried out, and protect against the worst-case scenario epidemics. The approaches we use here improve the resolution at which plant disease management strategies can be optimised compared with previous studies \citep[c.f.][]{forster_optimizing_2007}, and allow for more detail in the underlying spread model \citep[c.f.][]{epanchin_optimal_2012}.